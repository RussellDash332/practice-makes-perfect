\input ../../setup

\usepackage{hyperref}
\usepackage{listings}
\usepackage{color}
\usepackage{enumerate}
\usepackage{graphicx}
\usepackage{tikz-qtree}
\usepackage{amsmath}
\usepackage{amssymb}
\usepackage{multicol}
\usepackage{xcolor}
\setlength{\columnsep}{5mm}

% \renewcommand{\labelenumi}{\alph{enumi}.}
% \renewcommand{\labelenumii}{(\roman{enumii})}

\definecolor{mygreen}{rgb}{0,0.6,0}
\definecolor{mygray}{rgb}{0.5,0.5,0.5}
\definecolor{mymauve}{rgb}{0.58,0,0.82}


\lstdefinestyle{normalPy}{
language=Python,				% the language of the code
basicstyle=\footnotesize,			% the size of the fonts that are used for the code
numbers=left,				% where to put the line-numbers; possible values are (none, left, right)
numberstyle=\color{mygray},		% the style that is used for the line-numbers
stepnumber=1,				% the step between two line-numbers. If it's 1, each line will be numbered
numbersep=5pt,				% how far the line-numbers are from the code
backgroundcolor=\color{white},		% choose the background color; you must add \usepackage{color} or \usepackage{xcolor}
showstringspaces=false,			% underline spaces within strings only
showspaces=false,				% show spaces everywhere adding particular underscores; it overrides 'showstringspaces'
showtabs=false,				% show tabs within strings adding particular underscores
frame=shadowbox,				% adds a frame around the code
tabsize=4,					% sets default tabsize to 4 spaces
captionpos=t,				% sets the caption-position to bottom
breaklines=true,				% sets automatic line breaking
breakatwhitespace=false,			% sets if automatic breaks should only happen at whitespace
commentstyle=\color{mygreen},    	% comment style
keepspaces=true,                 		% keeps spaces in text, useful for keeping indentation of code (possibly needs columns=flexible)
keywordstyle=\color{blue},       		% keyword style
}

\lstdefinestyle{consolePy}{
stepnumber=0,
}
\tikzset{every tree node/.style={minimum width=2em,draw,circle},
         blank/.style={draw=none},
         edge from parent/.style=
         {draw, edge from parent path={(\tikzparentnode) -- (\tikzchildnode)}},
         level distance=1.5cm}

\begin{document}
%%%%%%%%%%%%%%%%%%%%%%%%%%%%%%%%%%%%%%%%%%%%%%%
\pmockheader{Mock Midterm}{}
%%%%%%%%%%%%%%%%%%%%%%%%%%%%%%%%%%%%%%%%%%%%%%%

\medskip

You have only \textbf{1 hour and 45 minutes} to solve all \textbf{FOUR (4) questions}. \\
The maximum attainable score is \textbf{75}. Good luck! \\ \\
\textit{Prepared by Russell Saerang.}

\begin{multicols*}{2}
[
\section{Question 1: Python Expressions [25 marks]}
There are several parts to this problem. Answer each part \underline{\textbf{independently and separately}}. \\
In each part, the Python snippet is entered into a Python script and then run. Determine the
response printed by the interpreter (Python shell) and \textbf{write the exact output}. 
If the interpreter produces an error message, or enters an infinite loop, explain why and
\textbf{clearly state the responsible evaluation step}.
]
\begin{enumerate}
\item[\textbf{A.}] 
\begin{python}
x, y = 3, 12
def f(x, y):
    x += y
    y %= x
    return x * y
z, y = f(y, x), 1
print(f(y, z - x))
\end{python}
\begin{flushright}
    [5 marks]
\end{flushright}

\item[\textbf{B.}]
\begin{python}
p, q = (), (7, 4, 2, 5, 3)
for r in q:
    if r % 5 == 2:
        p = (-1, r,) + p[1:]
    elif r % 2 == 0:
        p = (r,) + p + (r, -1)
print(p)
\end{python}
\begin{flushright}
    [5 marks]
\end{flushright}

\item[\textbf{C.}]
\begin{python}
m, p = "mutton", "python"
mp = m+p+"ton"
if "on" in mp:
    if p in mp:
        p = "p"
        print("yum")
    if "on" in p:
        print("bum")
    elif mp:
        print("tum")
\end{python}
\begin{flushright}
    [5 marks]
\end{flushright}

\item[\textbf{D.}]
\begin{python}
n = 420
while n >= -1:
    print(n)
    n //= 3
    if n % 2 == 0:
        print(2*n)
        continue
    elif n % 7 == 1:
        print(n/2)
        break
    n //= 2
    n -= 2
\end{python}
\begin{flushright}
    [5 marks]
\end{flushright}

\item[\textbf{E.}]
\begin{python}
def con(x, y):
    return lambda z: y(x, z)
def fuse(x):
    return lambda y: x(y, y)
why = lambda x,y: x*y
so = lambda x,y: x**2 + y
hard = lambda y,x: 2*x + y
dif = con(fuse(so)(3), why)
print(dif(fuse(hard)(2)))
\end{python}
\begin{flushright}
    [5 marks]
\end{flushright}
\end{enumerate}
\end{multicols*}

\newpage
\section{Question 2: Binomial Distribution [19 marks]}
Binomial distribution is a commonly used probability distribution, especially in the field of statistics 
and probability. It is a type of discrete probability distribution of a random variable $X$ where $X$ 
is the number of successes in a series of trials with two given parameters, $n$ and $p$, the 
number of trials and the probability of succeeding a single trial. \\ \\
Mathematically speaking, suppose $X$ is a random variable with parameters $n$ and $p$. We can 
write this as
\[X \sim Bin(n,p)\]
and the probability of having $k$ successes in $n$ trials can be denoted as
\[P(X = k) = \binom{n}{k}p^k(1-p)^{n-k}\]
Suppose we are to implement a function \colorbox{CornflowerBlue!20}{\texttt{choose}} that takes in two nonnegative integer inputs $n, k$
and returns
\[\binom{n}{k} = \frac{n!}{k!(n-k)!}\]
i.e. the number of ways to pick $k$ items out of $n$ items. \\
For the following questions you may assume that $k \le n$.
\begin{enumerate}
\item[\textbf{A.}]
Provide a \underline{recursive} implementation of the function \colorbox{CornflowerBlue!20}{\texttt{choose}}. For example, 
\texttt{\bfseries choose(10, 3)} will return \texttt{\bfseries 120}. \textbf{Note that you are not allowed to use any additional functions}.\\
\textbf{Hint:} Divide into two cases whether you picked the $k$-th item or not. How many items left to choose for each cases?
\begin{flushright}
    [4 marks]
\end{flushright}
\item[\textbf{B.}]
State the order of growth in terms of time and space for the function you wrote in Part (A). Briefly explain your answer.
\begin{flushright}
    [3 marks]
\end{flushright}
\item[\textbf{C.}]
Provide an \underline{iterative} implementation of the function \colorbox{CornflowerBlue!20}{\texttt{choose}}. \textbf{Note that you are not 
allowed to use any additional functions}.\\
\textbf{Hint:} Note that you can express $\binom{n}{k}$ like the following.
\[\binom{n}{k} = \frac{n!}{k!(n-k)!} = \frac{n \cdot (n-1) \cdot \ldots \cdot (n-k+1)}{1 \cdot 2 \cdot \ldots \cdot k}\]
\begin{flushright}
    [4 marks]
\end{flushright}
\item[\textbf{D.}]
State the order of growth in terms of time and space for the function you wrote in Part (C). Briefly explain your answer.
\begin{flushright}
    [2 marks]
\end{flushright}
\item[\textbf{E.}]
Now that we have defined the function \colorbox{CornflowerBlue!20}{\texttt{choose}}, we can find the probability that the random variable $X$ has 
a value less than or equal to some nonnegative integer $k$. Mathematically,
\begin{align*}
    P(X \le k) &= P(X = 0) + P(X = 1) + \ldots + P(X = k) \\
    &= \binom{n}{0}p^0(1-p)^{n-0} + \binom{n}{1}p^1(1-p)^{n-1} + \ldots + \binom{n}{k}p^k(1-p)^{n-k}
\end{align*}
The function \colorbox{CornflowerBlue!20}{\texttt{cumulative\_prob}} takes in a nonnegative integer for the number of trials, a real number as 
probability, and an integer representing the number of successes. It returns the cumulative probability of the random variable $X$ being 
less than or equal to the number of successes. \\ \\
For example, \texttt{\bfseries cumulative\_prob(10, 0.5, 3)} returns \texttt{\bfseries 0.171875} because it computes the value of $P(X \le 3)$ 
where $X$ is a binomial random variable obtained from 10 trials and success probability of 0.5.
\begin{align*}
    P(X \le 3) &= P(X = 0) + P(X = 1) + P(X = 2) + P(X = 3) \\
    &= \binom{10}{0}{0.5}^0(1-0.5)^{10-0} + \binom{10}{1}{0.5}^1(1-0.5)^{10-1} \\
    &\hspace{1cm} + \binom{10}{2}{0.5}^2(1-0.5)^{10-2} + \binom{10}{3}{0.5}^3(1-0.5)^{10-3} \\
    &= \left(\binom{10}{0} + \binom{10}{1} + \binom{10}{2} + \binom{10}{3}\right)(0.5)^{10}s \\
    &= (1 + 10 + 45 + 120)(0.5)^{10} \\
    &= \frac{176}{1024} = \fbox{0.171875}
\end{align*}
Using the previously defined functions, provide an implementation to the function \colorbox{CornflowerBlue!20}{\texttt{cumulative\_prob}}. You may 
assume that the given input will be a valid input, e.g. the probability given will lie in the $[0,1]$ interval.
\begin{flushright}
    [4 marks]
\end{flushright}
\item[\textbf{F.}]
Is your implementation in Part (E) recursive or iterative? State the order of growth in terms
of time and space of your implementation. Briefly explain your answer.
\begin{flushright}
    [2 marks]
\end{flushright}
\end{enumerate}

\newpage
\section{Question 3: Higher Order Distribution [10 marks]}
\begin{enumerate}
\item[\textbf{A.}]
\textbf{[HARD]} Consider the higher-order function \colorbox{CornflowerBlue!20}{\texttt{fold}} which was taught in class.
\begin{python}
def fold(op, f, n):
    if n == 0:
        return f(0)
    else:
        return op(f(n), fold(op, f, n-1))
\end{python}
The function \colorbox{CornflowerBlue!20}{\texttt{choose}} defined in Question 2 can be defined in terms of \colorbox{CornflowerBlue!20}{\texttt{fold}} 
as follows:
\begin{python}
def choose(n, k):
    <PRE>
    return int(fold(<T1>, <T2>, <T3>))
\end{python}
Please provide possible implementations for the terms \texttt{\bfseries T1}, \texttt{\bfseries T2}, and \texttt{\bfseries T3}. 
You may optionally define other functions in \texttt{\bfseries PRE} if needed. \\ \\
Note: You are to use the higher-order function and not solve it recursively or iteratively or use a formula.
\begin{flushright}
    [6 marks]
\end{flushright}

\item[\textbf{B.}]
Consider the higher-order function \colorbox{CornflowerBlue!20}{\texttt{sum}} which was taught in class.
\begin{python}
def sum(term, a, next, b):
    if a > b:
        return 0
    else:
        return term(a) + sum(term, next(a), next, b)
\end{python}
It turns out that the function \colorbox{CornflowerBlue!20}{\texttt{cumulative\_prob}} defined in Question 2 can also be defined in terms 
of \colorbox{CornflowerBlue!20}{\texttt{sum}} as follows:
\begin{python}
def cumulative_prob(n, p, k):
    <PRE2>
    return sum(<T4>, <T5>, <T6>, <T7>)
\end{python}
Please provide possible implementations for the terms \texttt{\bfseries T4}, \texttt{\bfseries T5}, \texttt{\bfseries T6}, 
and \texttt{\bfseries T7}. You may optionally define other functions in \texttt{\bfseries PRE2} if needed. \\ \\
Note: You are to use the higher-order function and not solve it recursively or iteratively or use a formula.
\begin{flushright}
    [4 marks]
\end{flushright} 
\end{enumerate}

\newpage
\section{Question 4: Brawl Stars! [21 marks]}
\textbf{INSTRUCTIONS: Please read the entire question clearly before you attempt this problem!! 
You are also not to use any Python data types which have not yet been taught in class.} \\ \\
\textbf{Background [OK to skip]} Brawl Stars is a multiplayer online battle arena and third-person hero shooter video 
game developed and published by the Finnish video game company Supercell. It was released worldwide on December 12, 2018 
on iOS and Android. The game features various game modes, each with a different objective. Players can choose from a 
selection of Brawlers, which are characters that can be controlled with on-screen joysticks in a game match. In Brawl 
Stars, players battle against other players or AI opponents in multiple game modes. Players can choose between characters 
called Brawlers that they have unlocked through Boxes, the Brawl Pass, the Trophy Road, or purchased through the Shop to 
use in battles.
\begin{flushright}
    \textit{Source: Wikipedia}
\end{flushright}

\vspace{5 mm}

You are so addicted to Brawl Stars you decided to implement a Brawl Stars account in Python. You hope that implementing such thing 
can help you study CS1010S and play Brawl Stars at the same time. \\ \\
Your first step is to design an account data type that stores the amount of coins the account currently has and the brawlers 
(Brawl Stars characters) unlocked along with the power points obtained for each brawler. Your account data type should support the 
following functions:

\begin{itemize}
\item \colorbox{CornflowerBlue!20}{\texttt{make\_account}} takes in no parameters and returns a new fresh account with 1000 coins at the start.
\item \colorbox{CornflowerBlue!20}{\texttt{add\_coins(account, amt)}} returns a new account with \texttt{\bfseries amt} coins added to the account.
\item \colorbox{CornflowerBlue!20}{\texttt{obtain\_powerpoint(account, brawler, points)}} returns a new account with \texttt{\bfseries points} power 
points added to \texttt{\bfseries brawler}'s statistics. If \texttt{\bfseries brawler} hasn't been unlocked before, add it to the collection. 
However, \textbf{2 coins} are deducted for each power point.
\item \colorbox{CornflowerBlue!20}{\texttt{get\_coins(account)}} returns the amount of coins the \texttt{\bfseries account} currently has.
\item \colorbox{CornflowerBlue!20}{\texttt{get\_brawlers(account)}} returns a tuple of all the existing brawlers' names from the \texttt{\bfseries account}.
\item \colorbox{CornflowerBlue!20}{\texttt{get\_level(account, brawler)}} returns the level of the \texttt{\bfseries brawler} in the 
\texttt{\bfseries account} based on the following table. If the \texttt{\bfseries brawler} does not exist in the \texttt{\bfseries account}, assume it has 
0 power points.
\end{itemize}

\begin{center}
\begin{tabular}{||c|c||} 
\hline
\textbf{Level} & \textbf{Points} \\ [0.5ex] 
\hline\hline
0 & 0 \\ 
\hline
1 & 1-19 \\ 
\hline
2 & 20-49 \\
\hline
3 & 50-99 \\
\hline
4 & 100-199 \\
\hline
5 & 200-549 \\
\hline
6 & 550 \\
\hline
\end{tabular}
\end{center}

Example execution:
\begin{python}
>>> acc = make_account()
>>> acc2 = add_coins(acc, 300)
>>> acc3 = obtain_powerpoint(acc2, "Shelly", 125)
>>> get_coins(acc3)
1050
>>> get_level(acc3, "Shelly")
4

>>> acc4 = obtain_powerpoint(acc3, "Colt", 300)
>>> acc5 = obtain_powerpoint(acc4, "Shelly", 75)
>>> get_brawlers(acc5)
('Shelly', 'Colt')

>>> get_level(acc5, "Shelly")
5
>>> get_level(acc5, "Poco")
0
\end{python}

\begin{enumerate}
\item[\textbf{A.}]
Decide on an implementation for the account object and implement \colorbox{CornflowerBlue!20}{\texttt{make\_account}}. 
Describe how the state is stored in your implementation for \texttt{\bfseries acc5}.
\begin{flushright}
    [3 marks]
\end{flushright}

\item[\textbf{B.}]
Implement the function \colorbox{CornflowerBlue!20}{\texttt{add\_coins(account, amt)}}.
\begin{flushright}
    [2 marks]
\end{flushright}

\item[\textbf{C.}]
Implement the function \colorbox{CornflowerBlue!20}{\texttt{obtain\_powerpoint(account, brawler, points)}}. You may assume that 
the power points that you have to pay for will not exceed your current coins balance.
\begin{flushright}
    [4 marks]
\end{flushright}

\item[\textbf{D.}]
Implement the function \colorbox{CornflowerBlue!20}{\texttt{get\_coins(account)}}.
\begin{flushright}
    [2 marks]
\end{flushright}

\item[\textbf{E.}]
Implement the function \colorbox{CornflowerBlue!20}{\texttt{get\_brawlers(account)}}. The alphabetic order of the brawlers is not important.
\begin{flushright}
    [3 marks]
\end{flushright}

\item[\textbf{F.}]   
Implement the function \colorbox{CornflowerBlue!20}{\texttt{get\_level(account, brawler)}}.
\begin{flushright}
    [3 marks]
\end{flushright}

\item[\textbf{G.}]   
\textbf{[Cash Overflow!]} It turns out that at some point, you might have more than 550 power points in a single brawler. 
You want to prevent a brawler from having more than 550 points, so the excess power points will simply be converted back into 
\textbf{2 coins} for each power point. For example, if you initially have a brawler with 400 power points and you obtain another 200 
power points, you will max out that brawler and get 100 extra coins as a "refund". \\ \\
Modify the \colorbox{CornflowerBlue!20}{\texttt{obtain\_powerpoint}} function such that it can cater to this problem. Again, you may assume that 
the power points that you have to pay for will not exceed your current coins balance.
\begin{flushright}
    [4 marks]
\end{flushright}
\end{enumerate}

\newpage
\section{Appendix}
The following are some functions that were introduced in class. For your reference, they are
reproduced here.
\begin{python}
def sum(term, a, next, b):
    if a > b:
        return 0
    else:
        return term(a) + sum(term, next(a), next, b)

def product(term, a, next, b):
    if a > b:
        return 1
    else:
        return term(a) * product(term, next(a), next, b)

def fold(op, f, n):
    if n == 0:
        return f(0)
    else:
        return op(f(n), fold(op, f, n-1))

def enumerate_interval(low, high):
    return tuple(range(low,high+1))

def map(fn, seq):
    if seq == ():
        return ()
    else:
        return (fn(seq[0]),) + map(fn, seq[1:])

def filter(pred, seq):
    if seq == ():
        return ()
    elif pred(seq[0]):
        return (seq[0],) + filter(pred, seq[1:])
    else:
        return filter(pred, seq[1:])

def accumulate(fn, initial, seq):
    if seq == ():
        return initial
    else:
        return fn(seq[0], accumulate(fn, initial, seq[1:]))
\end{python}

\end{document}