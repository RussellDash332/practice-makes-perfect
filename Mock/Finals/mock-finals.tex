\input ../../setup

\usepackage{hyperref}
\usepackage{listings}
\usepackage{color}
\usepackage{enumerate}
\usepackage{graphicx}
\usepackage{tikz-qtree}
\usepackage{amsmath}
\usepackage{amssymb}
\usepackage{multicol}
\usepackage{xcolor}
\setlength{\columnsep}{5mm}

% \renewcommand{\labelenumi}{\alph{enumi}.}
% \renewcommand{\labelenumii}{(\roman{enumii})}

\definecolor{mygreen}{rgb}{0,0.6,0}
\definecolor{mygray}{rgb}{0.5,0.5,0.5}
\definecolor{mymauve}{rgb}{0.58,0,0.82}


\lstdefinestyle{normalPy}{
language=Python,				% the language of the code
basicstyle=\footnotesize,			% the size of the fonts that are used for the code
numbers=left,				% where to put the line-numbers; possible values are (none, left, right)
numberstyle=\color{mygray},		% the style that is used for the line-numbers
stepnumber=1,				% the step between two line-numbers. If it's 1, each line will be numbered
numbersep=5pt,				% how far the line-numbers are from the code
backgroundcolor=\color{white},		% choose the background color; you must add \usepackage{color} or \usepackage{xcolor}
showstringspaces=false,			% underline spaces within strings only
showspaces=false,				% show spaces everywhere adding particular underscores; it overrides 'showstringspaces'
showtabs=false,				% show tabs within strings adding particular underscores
frame=shadowbox,				% adds a frame around the code
tabsize=4,					% sets default tabsize to 4 spaces
captionpos=t,				% sets the caption-position to bottom
breaklines=true,				% sets automatic line breaking
breakatwhitespace=false,			% sets if automatic breaks should only happen at whitespace
commentstyle=\color{mygreen},    	% comment style
keepspaces=true,                 		% keeps spaces in text, useful for keeping indentation of code (possibly needs columns=flexible)
keywordstyle=\color{blue},       		% keyword style
}

\lstdefinestyle{consolePy}{
stepnumber=0,
}
\tikzset{every tree node/.style={minimum width=2em,draw,circle},
         blank/.style={draw=none},
         edge from parent/.style=
         {draw, edge from parent path={(\tikzparentnode) -- (\tikzchildnode)}},
         level distance=1.5cm}

\begin{document}
%%%%%%%%%%%%%%%%%%%%%%%%%%%%%%%%%%%%%%%%%%%%%%%
\pmockheader{Mock Finals}{}
%%%%%%%%%%%%%%%%%%%%%%%%%%%%%%%%%%%%%%%%%%%%%%%

\medskip

You have only \textbf{1 hour and 30 minutes} to solve all \textbf{FOUR (4) questions}. \\
The maximum attainable score is \textbf{80}. Good luck! \\ \\
Prepared by Russell Saerang.

\begin{multicols*}{2}
[
\section{Question 1: Python Expressions [25 marks]}
There are several parts to this problem. Answer each part \underline{\textbf{independently and separately}}. \\ \\
In each part, the Python snippet is entered into a Python script and then run. Determine the
response printed by the interpreter (Python shell) and \textbf{write the exact output}. 
If the interpreter produces an error message, or enters an infinite loop, explain why and
\textbf{clearly state the responsible evaluation step}.
]
\begin{enumerate}
\item[\textbf{A.}] 
\begin{python}
Yo
\end{python}
\begin{flushright}
    [5 marks]
\end{flushright}

\item[\textbf{B.}]
\begin{python}
"foo"
\end{python}
\begin{flushright}
    [5 marks]
\end{flushright}

\item[\textbf{C.}]
\begin{python}
bin
\end{python}
\begin{flushright}
    [5 marks]
\end{flushright}

\item[\textbf{D.}]
\begin{python}
NameError
\end{python}
\begin{flushright}
    [5 marks]
\end{flushright}

\item[\textbf{E.}]
\begin{python}
print
\end{python}
\begin{flushright}
    [5 marks]
\end{flushright}
\end{enumerate}
\end{multicols*}

\newpage
\section{Question 2}
\begin{enumerate}
\item[\textbf{A.}]
\end{enumerate}

\newpage
\section{Question 3}
\begin{enumerate}
\item[\textbf{A.}]
\end{enumerate}

\newpage
\section{Question 4}
\textbf{INSTRUCTIONS: Please read the entire question clearly before you attempt this problem!! 
You are also not to use any Python data types which have not yet been taught in class.}
\begin{enumerate}
\item[\textbf{A.}]
\end{enumerate}

\newpage
\section{Appendix}

\end{document}