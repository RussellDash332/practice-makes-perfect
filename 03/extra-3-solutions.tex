\input ../setup

\usepackage{hyperref}
\usepackage{listings}
\usepackage{color}
\usepackage{enumerate}
\usepackage{graphicx}
\usepackage{tikz-qtree}

% \renewcommand{\labelenumi}{\alph{enumi}.}
% \renewcommand{\labelenumii}{(\roman{enumii})}

\definecolor{mygreen}{rgb}{0,0.6,0}
\definecolor{mygray}{rgb}{0.5,0.5,0.5}
\definecolor{mymauve}{rgb}{0.58,0,0.82}


\lstdefinestyle{normalPy}{
language=Python,				% the language of the code
basicstyle=\footnotesize,			% the size of the fonts that are used for the code
numbers=left,				% where to put the line-numbers; possible values are (none, left, right)
numberstyle=\color{mygray},		% the style that is used for the line-numbers
stepnumber=1,				% the step between two line-numbers. If it's 1, each line will be numbered
numbersep=5pt,				% how far the line-numbers are from the code
backgroundcolor=\color{white},		% choose the background color; you must add \usepackage{color} or \usepackage{xcolor}
showstringspaces=false,			% underline spaces within strings only
showspaces=false,				% show spaces everywhere adding particular underscores; it overrides 'showstringspaces'
showtabs=false,				% show tabs within strings adding particular underscores
frame=shadowbox,				% adds a frame around the code
tabsize=4,					% sets default tabsize to 4 spaces
captionpos=t,				% sets the caption-position to bottom
breaklines=true,				% sets automatic line breaking
breakatwhitespace=false,			% sets if automatic breaks should only happen at whitespace
commentstyle=\color{mygreen},    	% comment style
keepspaces=true,                 		% keeps spaces in text, useful for keeping indentation of code (possibly needs columns=flexible)
keywordstyle=\color{blue},       		% keyword style
}

\lstdefinestyle{consolePy}{
stepnumber=0,
}
\tikzset{every tree node/.style={minimum width=2em,draw,circle},
         blank/.style={draw=none},
         edge from parent/.style=
         {draw, edge from parent path={(\tikzparentnode) -- (\tikzchildnode)}},
         level distance=1.5cm}

\begin{document}
%%%%%%%%%%%%%%%%%%%%%%%%%%%%%%%%%%%%%%%%%%%%%%%
\psetheader{Extra Practice 3 Solutions}{}
%%%%%%%%%%%%%%%%%%%%%%%%%%%%%%%%%%%%%%%%%%%%%%%

\medskip

\section{Help}
Yes, we put these functions here so that you can refer to them easily.
\begin{python}
def sum(term, a, next, b):
    if a > b:
        return 0
    else:
        return term(a) + sum(term, next(a), next, b)

def fold(op, f, n):
    if n == 0:
        return f(0)
    else:
        return op(f(n), fold(op, f, n-1))
\end{python}

\section{Order of Growth}
Determine the time and space complexity of all these functions.
\begin{enumerate}[(a)]
\item
\begin{python}
def lol1(n, m):
    result = 0
    for i in range(n):
        for j in range(m):
            result += 1 
    return result
# Time: O(nm), Space: O(1)
\end{python}

\item
\begin{python}
def lol2(n):
    result = 0
    for i in range(n):
        for j in range(n):
            result += 1 
    return result
# Time: O(n**2), Space: O(1)
\end{python}

\item
\begin{python}
def lol3(n):
    result = ''
    for i in range(n):
        result += 'a' 
    return result
# Time: O(n**2), Space: O(n)
\end{python}

\newpage

\item
\begin{python}
def lol4(n):
    if n == 0:
        return 0
    else:
        return lol4(n - 1)
# Time: O(n), Space: O(n)
\end{python}

\item
\begin{python}
def lol5(n):
    result = 0
    for i in range(n):
        for j in range(i, n):
            result += 1 
    return result
# Time: O(n**2), Space: O(1)
\end{python}

\item
\begin{python}
def lol6(n):
    if n >= 1:
        return 0
    print("CS1010S is fun!")
    lol6(n // 2)
    lol6(n // 2)
# Time: O(1), Space: O(1)
\end{python}

\item
\begin{python}
def lol7(n):
    for i in range(n):
        for j in range(n + 1, i):
            print("Hello, I am Baymax")
# Time: O(n), Space: O(1)
\end{python}

\item
\begin{python}
def lol8(n):
    if n < 2:
        print("Less than two")
        return 1
    else:
        for j in range(1,n):
            print("CS1010S is fun!")
        a = lol8(n // 3)
        b = lol8(n // 3)
        c = lol8(n // 3)
        return a + b + c
# Time: O(n log n), Space: O(log n)
\end{python}

\item
\begin{python}
def lol9(n):
    if n <= 1:
        return
    print("CS1010S")
    for i in range(1, 2):
        lol9(n - 1)
# Time: O(n), Space: O(n)
\end{python}
\end{enumerate}

\newpage

\section{Higher Order Functions}
\begin{enumerate}[(a)]
\item Define a function \texttt{\bfseries total} that produces the output of the following code using either
\texttt{\bfseries sum} or \texttt{\bfseries fold}.
\begin{center}
\texttt{\bfseries 2 + 4 + 6 + 8 + 10}
\end{center}
\textbf{Solution:}
\begin{python}
# There are a lot of possibilities, here are the frequently used ones.
def total():
    return sum(lambda x: x, 2, lambda x: x + 2, 10)

def total():
    return sum(lambda x: 2*x, 1, lambda x: x + 1, 5)

def total():
    return fold(lambda x, y: x + y, lambda x: 2*x, 5)

def total():
    return fold(lambda x, y: 2*x + y, lambda x: x, 5)

def total():
    return fold(lambda x, y: x + y, lambda x: 2*x + 2, 5)

def total():
    return fold(lambda x, y: 2*x + y, lambda x: x + 1, 5)
\end{python}

\item I would like to convert a password such as "orange" into a string that comprises \textbf{only} of
"*", depending on how long my word is. This function will be named \texttt{\bfseries convert} and
take in a word string as an input while returning the converted word. You may assume
that the word will be at least one letter long. \\
\textbf{Sample Output:}
\begin{python}
>>> convert("orange")
'******'
>>> convert("ap13")
'****'
\end{python}
\begin{itemize}
\item Use an iterative approach to solve this. What is the time and space complexity? \\
\textbf{Solution:}
\begin{python}
# Actually Extra Practice 2
def convert(word):
    answer = ""
    for i in word:
        answer += "*"
    return answer

# Time: O(n**2) time complexity due to string concatenation
# Space: O(n) where n is the length of the word
\end{python}
\item Use a recursive approach to solve this. What is the time and space complexity? \\
\textbf{Solution:}
\begin{python}
def convert(word):
    if len(word) == 1:
        return "*"
    return "*" + convert(word[1:])
    # or return convert(word[:-1]) + "*"

# Time: O(n**2) due to string concatenation
# Space: O(n**2) where n is the length of the word
# (check recursion tree)
\end{python}
\item Use the \texttt{\bfseries fold} function to solve this. \\
\textbf{Solution:}
\begin{python}
def convert(word):
    return fold(lambda x, y: x + y, lambda x: "*", len(word) - 1)
\end{python}
\item Explain if the \texttt{\bfseries sum} function can be used to solve this. If not, explain what change needs
to be made to the original function and define it in terms of \texttt{\bfseries sum}. \\
\textbf{Solution:} \\
Note that the base case of the function is 0 but we are to return a string, so one suggestion is to change 0 on the 
base case to an empty string then we can do similar to what we did to our solution using \texttt{\bfseries fold}.
\end{itemize}

\item Now, I would like to filter out the letters "o" and "a" because I don't really like them. Define a
function \texttt{\bfseries remove} that takes in a word and returns the new word with all the "o"s and "a"s
removed.
\begin{python}
>>> remove("orange")
'rnge'
>>> remove("oooaaat")
't'
\end{python}
\begin{itemize}
\item Use an iterative approach to solve this. What is the time and space complexity? \\
\textbf{Solution:}
\begin{python}
def remove(word):
    answer = ""
    for i in word:
        if i != "o" and i != "a":
            answer += i
    return answer
\end{python}
\item Use a recursive approach to solve this. What is the time and space complexity? \\
\textbf{Solution:}
\begin{python}
def remove(word):
    if len(word) == 1:
        if word[0] == "o" or word[0] == "a":
            return ""
        return word
    if word[0] == "o" or word[0] == "a":
        return remove(word[1:])
    return word[0] + remove(word[1:])
\end{python}
\item Use the \texttt{\bfseries fold} function to solve this. \\
\textbf{Solution:}
\begin{python}
def remove(word):
    def is_oa(letter):
        return letter == "o" or letter == "a"
    return fold(lambda x, y: x + y,
                lambda x: "" if is_oa(word[-1-x]) else word[-1-x],
                len(word) - 1)
    # You can change the indexing from -1-x to len(word)-1-x
    # which makes more sense but longer to write
\end{python}
\item Explain if the \texttt{\bfseries sum} function can be used to solve this. If not, explain what change needs
to be made to the original function and define it in terms of \texttt{\bfseries sum}. \\
\textbf{Solution:} \\
Similar to above, note that the base case of the function is 0 but we are to return a string, so one suggestion is to change 0 on the 
base case to an empty string then we can do similar to what we did to our solution using \texttt{\bfseries fold}.
\item \textbf{\textit{How do you modify the functions to remove all vowels in a word?}} \\
\textbf{Solution:} \\
Simply add more conditionals such as
\begin{python}
i != "o" and i != "a" and i != "e" and i != "i" and i != "u"
\end{python}
Or alternatively, create a string and use the built-in function \texttt{\bfseries in} such as
\begin{python}
i not in "aieou"
\end{python}
\end{itemize}
\end{enumerate}

\begin{flushright}
\vspace{2 cm}\textbf{\textit{Solution compiled by Russell Saerang.}}
\end{flushright}
\end{document}