\input ../setup

\usepackage{hyperref}
\usepackage{listings}
\usepackage{color}
\usepackage{enumerate}
\usepackage{graphicx}
\usepackage{tikz-qtree}

% \renewcommand{\labelenumi}{\alph{enumi}.}
% \renewcommand{\labelenumii}{(\roman{enumii})}

\definecolor{mygreen}{rgb}{0,0.6,0}
\definecolor{mygray}{rgb}{0.5,0.5,0.5}
\definecolor{mymauve}{rgb}{0.58,0,0.82}


\lstdefinestyle{normalPy}{
language=Python,				% the language of the code
basicstyle=\footnotesize,			% the size of the fonts that are used for the code
numbers=left,				% where to put the line-numbers; possible values are (none, left, right)
numberstyle=\color{mygray},		% the style that is used for the line-numbers
stepnumber=1,				% the step between two line-numbers. If it's 1, each line will be numbered
numbersep=5pt,				% how far the line-numbers are from the code
backgroundcolor=\color{white},		% choose the background color; you must add \usepackage{color} or \usepackage{xcolor}
showstringspaces=false,			% underline spaces within strings only
showspaces=false,				% show spaces everywhere adding particular underscores; it overrides 'showstringspaces'
showtabs=false,				% show tabs within strings adding particular underscores
frame=shadowbox,				% adds a frame around the code
tabsize=4,					% sets default tabsize to 4 spaces
captionpos=t,				% sets the caption-position to bottom
breaklines=true,				% sets automatic line breaking
breakatwhitespace=false,			% sets if automatic breaks should only happen at whitespace
commentstyle=\color{mygreen},    	% comment style
keepspaces=true,                 		% keeps spaces in text, useful for keeping indentation of code (possibly needs columns=flexible)
keywordstyle=\color{blue},       		% keyword style
}

\lstdefinestyle{consolePy}{
stepnumber=0,
}
\tikzset{every tree node/.style={minimum width=2em,draw,circle},
         blank/.style={draw=none},
         edge from parent/.style=
         {draw, edge from parent path={(\tikzparentnode) -- (\tikzchildnode)}},
         level distance=1.5cm}

\begin{document}
%%%%%%%%%%%%%%%%%%%%%%%%%%%%%%%%%%%%%%%%%%%%%%%
\psetheader{Extra Practice 3}{}
%%%%%%%%%%%%%%%%%%%%%%%%%%%%%%%%%%%%%%%%%%%%%%%

\medskip

\section{Help}
Yes, we put this function here so that you can refer to it easily.
\begin{python}
def sum(term, a, next, b):
    if a > b:
        return 0
    else:
        return term(a) + sum(term, next(a), next, b)

def fold(op, f, n):
    if n == 0:
        return f(0)
    else:
        return op(f(n), fold(op, f, n-1))
\end{python}

\section{Order of Growth}
Determine the time and space complexity of all these functions.
\begin{enumerate}[(a)]
\item
\begin{python}
def lol1(n, m):
    result = 0
    for i in range(n):
        for j in range(m):
            result += 1 
    return result
\end{python}

\item
\begin{python}
def lol2(n):
    result = 0
    for i in range(n):
        for j in range(n):
            result += 1 
    return result
\end{python}

\item
\begin{python}
def lol3(n):
    result = ''
    for i in range(n):
        result += 'a' 
    return result
\end{python}

\newpage

\item
\begin{python}
def lol4(n):
    if n == 0:
        return 0
    else:
        return lol4(n - 1)
\end{python}

\item
\begin{python}
def lol5(n):
    result = 0
    for i in range(n):
        for j in range(i, n):
            result += 1 
    return result
\end{python}

\item
\begin{python}
def lol6(n):
    if n >= 1:
        return 0
    print("CS1010S is fun!")
    lol6(n // 2)
    lol6(n // 2)
\end{python}

\item
\begin{python}
def lol7(n):
    for i in range(n):
        for j in range(n + 1, i):
            print("Hello, I am Baymax")
\end{python}

\item
\begin{python}
def lol8(n):
    if n < 2:
        print("Less than two")
        return 1
    else:
        for j in range(1,n):
            print("CS1010S is fun!")
        a = lol8(n // 3)
        b = lol8(n // 3)
        c = lol8(n // 3)
        return a + b + c
\end{python}

\item
\begin{python}
def lol9(n):
    if n <= 1:
        return
    print("CS1010S")
    for i in range(1, 2):
        lol9(n - 1)
\end{python}
\end{enumerate}

\newpage

\section{Higher Order Functions}
\begin{enumerate}[(a)]
\item Define a function \texttt{\bfseries total} that produces the output of the following code using either
\texttt{\bfseries sum} or \texttt{\bfseries fold}.
\begin{center}
\texttt{\bfseries 2 + 4 + 6 + 8 + 10}
\end{center}

\item I would like to convert a password such as "orange" into a string that comprises \textbf{only} of
"*", depending on how long my word is. This function will be named \texttt{\bfseries convert} and
take in a word string as an input while returning the converted word. You may assume
that the word will be at least one letter long. \\
\textbf{Sample Output:}
\begin{python}
>>> convert("orange")
'******'
>>> convert("ap13")
'****'
\end{python}
\begin{itemize}
\item Use an iterative approach to solve this. What is the time and space complexity?
\item Use an iterative approach to solve this. What is the time and space complexity?
\item Use the \texttt{\bfseries fold} function to solve this.
\item Explain if the \texttt{\bfseries sum} function can be used to solve this. If not, explain what change needs
to be made to the original function and define it in terms of \texttt{\bfseries sum}.
\end{itemize}

\item Now, I would like to filter out the letters "o" and "a" because I don't really like them. Define a
function \texttt{\bfseries remove} that takes in a word and returns the new word with all the "o"s and "a"s
removed.
\begin{python}
>>> remove("orange")
'rnge'
>>> remove("oooaaat")
't'
\end{python}
\begin{itemize}
\item Use an iterative approach to solve this. What is the time and space complexity?
\item Use an iterative approach to solve this. What is the time and space complexity?
\item Use the \texttt{\bfseries fold} function to solve this.
\item Explain if the \texttt{\bfseries sum} function can be used to solve this. If not, explain what change needs
to be made to the original function and define it in terms of \texttt{\bfseries sum}.
\item \textbf{\textit{How do you modify the functions to remove all vowels in a word?}}
\end{itemize}
\end{enumerate}

\end{document}