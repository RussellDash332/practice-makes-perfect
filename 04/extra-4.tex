\input ../setup

\usepackage{hyperref}
\usepackage{listings}
\usepackage{color}
\usepackage{enumerate}
\usepackage{graphicx}
\usepackage{tikz-qtree}
\usepackage{amsmath}
\usepackage{amssymb}

% \renewcommand{\labelenumi}{\alph{enumi}.}
% \renewcommand{\labelenumii}{(\roman{enumii})}

\definecolor{mygreen}{rgb}{0,0.6,0}
\definecolor{mygray}{rgb}{0.5,0.5,0.5}
\definecolor{mymauve}{rgb}{0.58,0,0.82}


\lstdefinestyle{normalPy}{
language=Python,				% the language of the code
basicstyle=\footnotesize,			% the size of the fonts that are used for the code
numbers=left,				% where to put the line-numbers; possible values are (none, left, right)
numberstyle=\color{mygray},		% the style that is used for the line-numbers
stepnumber=1,				% the step between two line-numbers. If it's 1, each line will be numbered
numbersep=5pt,				% how far the line-numbers are from the code
backgroundcolor=\color{white},		% choose the background color; you must add \usepackage{color} or \usepackage{xcolor}
showstringspaces=false,			% underline spaces within strings only
showspaces=false,				% show spaces everywhere adding particular underscores; it overrides 'showstringspaces'
showtabs=false,				% show tabs within strings adding particular underscores
frame=shadowbox,				% adds a frame around the code
tabsize=4,					% sets default tabsize to 4 spaces
captionpos=t,				% sets the caption-position to bottom
breaklines=true,				% sets automatic line breaking
breakatwhitespace=false,			% sets if automatic breaks should only happen at whitespace
commentstyle=\color{mygreen},    	% comment style
keepspaces=true,                 		% keeps spaces in text, useful for keeping indentation of code (possibly needs columns=flexible)
keywordstyle=\color{blue},       		% keyword style
}

\lstdefinestyle{consolePy}{
stepnumber=0,
}
\tikzset{every tree node/.style={minimum width=2em,draw,circle},
         blank/.style={draw=none},
         edge from parent/.style=
         {draw, edge from parent path={(\tikzparentnode) -- (\tikzchildnode)}},
         level distance=1.5cm}

\begin{document}
%%%%%%%%%%%%%%%%%%%%%%%%%%%%%%%%%%%%%%%%%%%%%%%
\psetheader{Extra Practice 4}{}
%%%%%%%%%%%%%%%%%%%%%%%%%%%%%%%%%%%%%%%%%%%%%%%

\medskip

\section{Help}
Yes, we put these functions here so that you can refer to them easily.
\begin{python}
def sum(term, a, next, b):
    if a > b:
        return 0
    else:
        return term(a) + sum(term, next(a), next, b)

def fold(op, f, n):
    if n == 0:
        return f(0)
    else:
        return op(f(n), fold(op, f, n-1))

def compose(f, g):
    return lambda x: f(g(x))

def thrice(f):
    return compose(compose(f, f), f)
\end{python}

\section{Question 1}
This question tests you about the left-right evaluation.
\begin{python}
def new_if(predicate, then, otherwise):
    if predicate:
        then
    else:
        otherwise

def p(x):
    new_if(x > 5, print(x), p(x+1))

p(1) # what happens here?
\end{python}

\newpage

\section{Question 2}
It's time for a simple function nesting! What's the output of the code below?
\begin{python}
print((lambda x: lambda y: 2*x)(3)(4))
\end{python}

\section{Question 3}
Let's apply the same idea from Mission 3 :D
\begin{python}
print(thrice(thrice)(lambda x: x-1)(27))
\end{python}

\section{Question 4}
Nested function calls. Can you do it?
\begin{python}
foo = lambda y: lambda x: x(y)
add1 = lambda x: x+1

print(foo(add1(2))(foo)(add1))
\end{python}

\section{Question 5}
Nested variable scopes. Can you do it?
\begin{python}
def foo(x):
    def bar(x, y):
        return lambda y: y(x)
    return lambda y: bar(x, y)

print(foo(lambda x: x**3)(lambda x: x**2)(lambda x: x)(4))
\end{python}

\section{Question 6}
Try to do each subquestion within \textbf{5 minutes}. \underline{You are to use the \texttt{\bfseries fold}
function} in all of the following questions.
\begin{enumerate}
\item Define a function \texttt{\bfseries between} that takes in a word as an input and returns a new
string with "*" placed in between all consecutive words that are identical. \\
\textbf{Sample Tests:}
\begin{python}
>>> between("happy")
'hap*py'
>>> between("oookayy")
'o*o*okay*y'
\end{python}

\item Define a function \texttt{\bfseries check\_vowel} that takes in a word as an input and returns
\texttt{\bfseries True} if there is at least one vowel in the word or \texttt{\bfseries False} otherwise. This is
case-sensitive. \\
\newpage
\textbf{Sample Tests:}
\begin{python}
>>> check_vowel("qwertyjkl")
True
>>> check_vowel("482jfn")
False
\end{python}

\item We have a tuple that contains some integers. Define a function \texttt{\bfseries largest} that
takes in a tuple and returns the largest integer. \\
\textbf{Sample Tests:}
\begin{python}
>>> largest((4, 2, 6, 2, 1))
6
>>> largest((0, 0, 0, 0, 0))
0
\end{python}
\end{enumerate}

\section{Question 7}
Given a tuple containing numbers, find the number of combination of numbers that
you can use from the tuple that can add up to a target number. Define this function
\texttt{\bfseries no\_of\_ways} that takes in a tuple and a target number, and return the number of ways
you can hit this target. You can only use each number once. You do not need to use
any higher-order functions for this. \\ \\
\textbf{Sample Tests:}
\begin{python}
>>> no_of_ways((4, 2, 5), 8)
0                               # no way you can make the number 8
>>> no_of_ways((4, 2, 5, 3), 7)
2                               # either 2 + 5 or 4 + 3 gives 7
>>> no_of_ways((4, 2, 5, 3, 5, 1), 5)
4                               # either 4 + 1, 2 + 3, 5 or 5 gives 5
\end{python}

\section{Question 8}
Define a function that returns the sum of cubes for integers from 1 to n, using \texttt{\bfseries accumulate}.
Recall the definition of \texttt{\bfseries accumulate} below.
\[a_1 = a, a_n \le b\]
\[\text{accumulate}(\oplus, \text{base}, f, a, \text{next}, b): (f(a_1) \oplus (f(a_2) \oplus (... \oplus (f(a_n) \oplus \text{base})...)))\]

\section{Question 9}
Define the double factorial function using \texttt{\bfseries accumulate}.
\[
    n!! =
    \begin{cases}
        (n)(n-2)(n-4)\cdots(4)(2) & \text{if } \frac{n}{2} \in \mathbb{Z^+} \\
        (n)(n-2)(n-4)\cdots(3)(1) & \text{if } \frac{n+1}{2} \in \mathbb{Z^+}
    \end{cases}    
\]
\end{document}