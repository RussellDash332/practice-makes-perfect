\input ../setup

\usepackage{hyperref}
\usepackage{listings}
\usepackage{color}
\usepackage{enumerate}
\usepackage{graphicx}
\usepackage{tikz-qtree}
\usepackage{amsmath}
\usepackage{amssymb}
\usepackage{caption}

% \renewcommand{\labelenumi}{\alph{enumi}.}
% \renewcommand{\labelenumii}{(\roman{enumii})}

\definecolor{mygreen}{rgb}{0,0.6,0}
\definecolor{mygray}{rgb}{0.5,0.5,0.5}
\definecolor{mymauve}{rgb}{0.58,0,0.82}


\lstdefinestyle{normalPy}{
language=Python,				% the language of the code
basicstyle=\footnotesize,			% the size of the fonts that are used for the code
numbers=left,				% where to put the line-numbers; possible values are (none, left, right)
numberstyle=\color{mygray},		% the style that is used for the line-numbers
stepnumber=1,				% the step between two line-numbers. If it's 1, each line will be numbered
numbersep=5pt,				% how far the line-numbers are from the code
backgroundcolor=\color{white},		% choose the background color; you must add \usepackage{color} or \usepackage{xcolor}
showstringspaces=false,			% underline spaces within strings only
showspaces=false,				% show spaces everywhere adding particular underscores; it overrides 'showstringspaces'
showtabs=false,				% show tabs within strings adding particular underscores
frame=shadowbox,				% adds a frame around the code
tabsize=4,					% sets default tabsize to 4 spaces
captionpos=t,				% sets the caption-position to bottom
breaklines=true,				% sets automatic line breaking
breakatwhitespace=false,			% sets if automatic breaks should only happen at whitespace
commentstyle=\color{mygreen},    	% comment style
keepspaces=true,                 		% keeps spaces in text, useful for keeping indentation of code (possibly needs columns=flexible)
keywordstyle=\color{blue},       		% keyword style
}

\lstdefinestyle{consolePy}{
stepnumber=0,
}
\tikzset{every tree node/.style={minimum width=2em,draw,circle},
         blank/.style={draw=none},
         edge from parent/.style=
         {draw, edge from parent path={(\tikzparentnode) -- (\tikzchildnode)}},
         level distance=1.5cm}

\begin{document}
%%%%%%%%%%%%%%%%%%%%%%%%%%%%%%%%%%%%%%%%%%%%%%%
\psetheader{Extra Practice 8 (Questions Only)}{}
%%%%%%%%%%%%%%%%%%%%%%%%%%%%%%%%%%%%%%%%%%%%%%%

\medskip

\section{Question 1: Among Us}
In a game of "Among Us", there are two groups of characters - normal crewmates and
impostors. \\
\begin{center}
\includegraphics[scale=0.7]{amogus.png}
\end{center}
The rules for our version are:
\begin{itemize}
    \item Everyone starts off without a location and has to move to a location first (we'll assume
    this is not going to be a problem)
    \item Only the impostors can kill crewmates, and can only kill a crewmate if they are in the
    same location
    \item If a crewmate dies, he becomes a ghost and can still move around like a usual crewmate
    and do tasks
    \item Both impostors and crewmate can report a dead body, but must be in the same location
    as the dead body
    \item Impostors can pretend to do tasks as well
\end{itemize}

Define a class \texttt{\bfseries Place} that is initialized with a name. It should have the following methods:
\begin{itemize}
\item \texttt{\bfseries get\_name()} that returns the name of the place
\item \texttt{\bfseries get\_people()} that returns the names of all the people (dead or alive) in the
location currently in a string \texttt{\bfseries "[Person A], [Person B], [Person C] in
[name of place]"}. If there is no one, return \texttt{\bfseries "Nobody here"}
\end{itemize}

Define another class \texttt{\bfseries Person} that is initialized with a name. It should have the following
methods:
\begin{itemize}
\item \texttt{\bfseries get\_name()} that returns the name of the place
\item \texttt{\bfseries get\_state()} that returns \texttt{\bfseries "Alive"} if the person is still alive and 
\texttt{\bfseries "Killed by [Murderer's name]"} if he/she is dead
\item \texttt{\bfseries do\_task(task)} that takes in a task as an input string and returns the following
sentence \texttt{\bfseries "[Name of person] does [Name of task]."}
\item \texttt{\bfseries move(location)} that takes in a place object and moves the person from his current
place to that new location
\begin{itemize}
\item At the start of the game when this is called, return \texttt{\bfseries "[Name of person]
moves to [Name of place]"}
\item If the location is his current location, return \texttt{\bfseries "Already here"}
\item Otherwise, return \texttt{\bfseries "[Name of person] moves from [Current location name] to [New location name]"}
\end{itemize}
\item \texttt{\bfseries report(person)} that takes in a person object and tries to report it
\begin{itemize}
\item If the person reporting is already dead, return \texttt{\bfseries "Ghosts cannot report"}
\item If the person he is trying to report is oneself, return \texttt{\bfseries "Cannot report oneself"}
\item If the person he is trying to report is not in the same location as he is in, return
\texttt{\bfseries "[Name of other person] is in [Name of location]"}
\item If the person he is trying to report is still alive, return \texttt{\bfseries "[Name of person]
is still alive"}
\item Otherwise, return \texttt{\bfseries "[Name of person] reports [Name of other
person]..."} \\
He also immediately suspects everyone alive in the room at that moment and additionally returns the sentence 
\texttt{\bfseries "... and suspects [Person A's name], [Person B's name] ... [Person D's name]"} \\
If there is nobody else that he can suspect, the sentence \texttt{\bfseries "... and
nobody to suspect"} is returned
\end{itemize}
\end{itemize}

We now want to define another class \texttt{\bfseries Impostor} that is a subclass of \texttt{\bfseries Person}, and is
initialized with two inputs - name and weapon. It should have the following additional methods:
\begin{itemize}
\item \texttt{\bfseries kill(person)} that takes in a person object and attempts to kill it
\begin{itemize}
\item If the person is oneself, return \texttt{\bfseries "Cannot kill oneself"}
\item If the person is not in the same room as him, return \texttt{\bfseries "[Victim's name] is
in [Name of location victim is in]. Cannot kill"}
\item If the person is a fellow impostor, return \texttt{\bfseries "Cannot kill another impostor"}
\item If the person is already dead, return \texttt{\bfseries "Already killed"}
\item Otherwise, kill that person and return \texttt{\bfseries "[Name] killed [name of victim] with a [name of weapon]"}
\end{itemize}
\item \texttt{\bfseries victim\_list()} that returns the names of all the people he killed in alphabetical
order \texttt{\bfseries "[Name] killed [Person A], [Person B] ... [Person D]."}
\begin{itemize}
\item If he has not killed anyone yet, return \texttt{\bfseries "Killed nobody"}
\end{itemize}
\end{itemize}

\newpage

\section{Question 2: Discord Muting Saga}
\textbf{\textit{Background (very okay to skip):}} \\
\textit{
Discord is a VoIP, instant messaging and digital distribution platform designed for creating communities. 
Users communicate with voice calls, video calls, text messaging, media and files in private chats or as part 
of communities called "servers". Servers are a collection of persistent chat rooms and voice chat channels. \\ \\
Discord communities are organized into discrete collections of channels called servers. Servers are referred to as 
"guilds" in the developer documentation. Users can create servers for free, manage their public visibility and 
create both channels and channel categories up to 250. \\ \\
Channels may be either used for voice chat and streaming or for instant messaging and file sharing. The visibility 
and access to channels can be customized to limit access from certain users. Text channels support some rich text via 
a subset of the Markdown syntax. Code blocks with language-specific highlighting can also be used.
}
\begin{flushright}
    \textit{Source: Wikipedia}
\end{flushright}

\begin{figure}
    \begin{center}
        \includegraphics[scale=0.4]{carlbot.png}
    \end{center}
    \captionsetup{labelformat=empty}
    \caption{Carl-bot, possibly the most powerful turtle in the whole world}
\end{figure}

\textbf{\textit{The question itself}} \\
In our Discord server, you might have noticed the existence of Carl-bot, a bot that has the ability to manage autoroles 
and automoderations on a server. With this bot, you get to be assigned to your respective tutorial channels. \\ \\
However, Carl-bot has a feature that enables anyone with the server-managing permission to mute other members for any duration with any kinds of reasons. This is useful to moderate the server, but also dangerous as moderators can abuse their powers with the role they currently have. \\ \\
In this case, we decided to not care if this feature is dangerous or not. Instead, we can simulate how we can \textbf{abuse Carl-bot's muting capability.} \\ \\
In our "server", there are a few types of Discord users that we can implement, which is a regular user, a student, a tutor, or the owner of the server. There are also several channels which has a restricted access only for a list of roles. There are a few ground rules:
\begin{itemize}
    \item A user can only send messages in a channel if the user has joined the channel and is not muted.
    \item A user can only join channels if the user is not muted.
    \item Only the owner and the tutors can mute any user. However, if the tutor is not a server owner, he/she is not 
    allowed to mute a fellow tutor. The same rule applies for unmuting.
    \item A muted user cannot do anything. Either joining a channel, muting someone else, sending a message, or 
    unmuting someone else.
    \item There is no option to leave the server. Have fun in this chaos!
\end{itemize}

Define a class \texttt{\bfseries Channel} that is initialized with a name and a list of strings representing the 
permitted roles. However, the initialized name has no hashtag in front of it, so we have to add it when we initialize a new channel.
For example, from \texttt{\bfseries "name"} to \texttt{\bfseries "\#name"}. \\
It should have the following methods:
\begin{itemize}
    \item \texttt{\bfseries get\_name()} that returns the name of the channel
    \item \texttt{\bfseries get\_permitted\_roles()} that returns the list of all permitted roles
    \item \texttt{\bfseries get\_members()} that returns the list of all members who joined the channel, sorted alphabetically
    \item \texttt{\bfseries hall\_of\_mute()} that returns the list of all muted members who joined the channel, sorted alphabetically
\end{itemize}

Now define a class \texttt{\bfseries User} that is initialized with a name. Once a \texttt{\bfseries User} object is created, it has no role. It should have the following methods:
\begin{itemize}
    \item \texttt{\bfseries get\_role()} returns \texttt{\bfseries "[Name] is a [role]"} if the user has a role and 
    \texttt{\bfseries "[Name] has no role"} otherwise. However, if the user is the owner, return \texttt{\bfseries "[Name] is the owner!"}
    \item \texttt{\bfseries get\_channels()} returns the list of all channel names the user joined, sorted alphabetically
    \item \texttt{\bfseries join(channel)} that takes in a channel object and tries to join it
    \begin{itemize}
        \item If the user has already joined the channel, return \texttt{\bfseries "[Name] has already joined [Channel name]"}
        \item If the user is muted, return \texttt{\bfseries "[Name] is muted!"}. This is because a muted user cannot join a new channel
        \item If either the user is the owner of the server or the user's role is in the channel's permitted roles, return \texttt{\bfseries "[Name] joins [Channel name]"}
        \item Else, the user has no permission, return \texttt{\bfseries "[Name] has no permission to join [Channel name]"}
    \end{itemize}
    \item \texttt{\bfseries message(channel, msg)} that takes in a channel object and a message, then tries to send a message inside the channel
    \begin{itemize}
        \item If the user has not joined the channel, return \texttt{\bfseries "[Name] has not joined [Channel name]"}
        \item If the user is muted, return \texttt{\bfseries "[Name] is not allowed to send messages in [Channel name]"}
        \item Else, return a notification in the format of \texttt{\bfseries "[Channel name] --- [Name]: [Message]"}
    \end{itemize}
    \item \texttt{\bfseries mute(other, time, reason)} that takes in:
    \begin{itemize}
        \item \texttt{\bfseries other}: Another \texttt{\bfseries User} object that wants to be muted
        \item \texttt{\bfseries time}: The duration of the muting. If unspecified, it should be \texttt{\bfseries None}
        \item \texttt{\bfseries reason}: The reason of the muting, which will be a string
    \end{itemize}
    and tries to mute the user with the given details:
    \begin{itemize}
        \item If the target user is the same as the muter, return \texttt{\bfseries "Cannot mute oneself"}
        \item If the muter is muted, return \texttt{\bfseries "[Name] is not allowed to send messages!"}
        \item If the target user is already muted, return \texttt{\bfseries "[Other user's name] is muted!"}
        \item If the muter is the owner, mute the target user with the following rules:
        \begin{itemize}
            \item If the time is unspecified, return 
            \texttt{\bfseries "[Name] muted [Other user's name] indefinitely. Reason: [Reason]"}
            \item Else, return
            \texttt{\bfseries "[Name] muted [Other user's name] for [Time] minutes. Reason: [Reason]"}
        \end{itemize}
        \item If the muter is a tutor, the following rules apply:
        \begin{itemize}
            \item If the target user is a fellow tutor, return \texttt{\bfseries "Cannot mute a fellow tutor"}
            \item If the target user is the owner, the same thing will happen because the owner is also a tutor in this case
            \item Other than that, mute the target user with the same rule that applies when the owner is the muter
        \end{itemize}
        \item If none of the above applies, the user has no permission to mute other users, so return 
        \texttt{\bfseries "[Name] doesn't have a permission to mute another user"}
    \end{itemize}
    \item \texttt{\bfseries unmute(other)} that takes in a \texttt{\bfseries User} object and tries to unmute him/her
    \begin{itemize}
        \item If the unmuter tries to unmute oneself while being muted, return \texttt{\bfseries "[Name] is not allowed to send messages!"}
        \item If the target user is not muted, return \texttt{\bfseries "[Other user's name] is not muted :)"}
        \item If the unmuter is the owner, always unmute the target user and return 
        \texttt{\bfseries "[Name] unmuted [Other user's name]!"}
        \item If the unmuter is a tutor, the following rules apply:
        \begin{itemize}
            \item If the target user is also a tutor, return \texttt{\bfseries "Cannot unmute a fellow tutor"}
            \item Else, unmute the target user and return \texttt{\bfseries "[Name] unmuted [Other user's name]!"}
        \end{itemize}
        \item Else, the user has no permission to unmute other users, so return 
        \texttt{\bfseries "[Name] doesn't have a permission to unmute another user"}
    \end{itemize}
\end{itemize}

Next, define a class \texttt{\bfseries Tutor} that is a subclass of \texttt{\bfseries User}. The only difference between \texttt{\bfseries Tutor} 
and \texttt{\bfseries User} is that when being initialized, \texttt{\bfseries Tutor} will have the role \texttt{\bfseries "Tutor"} instead 
of \texttt{\bfseries None}. You may override the methods defined in the \texttt{\bfseries User} class if needed.

Similarly, define a class \texttt{\bfseries Student} that is a subclass of \texttt{\bfseries User}. The only difference between \texttt{\bfseries Student} 
and \texttt{\bfseries User} is that when being initialized, \texttt{\bfseries Tutor} will have the role \texttt{\bfseries "Student"} instead 
of \texttt{\bfseries None}. You may override the methods defined in the \texttt{\bfseries User} class if needed.

Finally, define a class \texttt{\bfseries Owner} that is also a \texttt{\bfseries Tutor}. However, the owner's role will be a list of three roles,
\texttt{\bfseries ["Owner", "Tutor", "Student"]}. You may override the methods defined in the \texttt{\bfseries User} class if needed.
\end{document}