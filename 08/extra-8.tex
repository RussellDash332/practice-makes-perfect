\input ../setup

\usepackage{hyperref}
\usepackage{listings}
\usepackage{color}
\usepackage{enumerate}
\usepackage{graphicx}
\usepackage{tikz-qtree}
\usepackage{amsmath}
\usepackage{amssymb}

% \renewcommand{\labelenumi}{\alph{enumi}.}
% \renewcommand{\labelenumii}{(\roman{enumii})}

\definecolor{mygreen}{rgb}{0,0.6,0}
\definecolor{mygray}{rgb}{0.5,0.5,0.5}
\definecolor{mymauve}{rgb}{0.58,0,0.82}


\lstdefinestyle{normalPy}{
language=Python,				% the language of the code
basicstyle=\footnotesize,			% the size of the fonts that are used for the code
numbers=left,				% where to put the line-numbers; possible values are (none, left, right)
numberstyle=\color{mygray},		% the style that is used for the line-numbers
stepnumber=1,				% the step between two line-numbers. If it's 1, each line will be numbered
numbersep=5pt,				% how far the line-numbers are from the code
backgroundcolor=\color{white},		% choose the background color; you must add \usepackage{color} or \usepackage{xcolor}
showstringspaces=false,			% underline spaces within strings only
showspaces=false,				% show spaces everywhere adding particular underscores; it overrides 'showstringspaces'
showtabs=false,				% show tabs within strings adding particular underscores
frame=shadowbox,				% adds a frame around the code
tabsize=4,					% sets default tabsize to 4 spaces
captionpos=t,				% sets the caption-position to bottom
breaklines=true,				% sets automatic line breaking
breakatwhitespace=false,			% sets if automatic breaks should only happen at whitespace
commentstyle=\color{mygreen},    	% comment style
keepspaces=true,                 		% keeps spaces in text, useful for keeping indentation of code (possibly needs columns=flexible)
keywordstyle=\color{blue},       		% keyword style
}

\lstdefinestyle{consolePy}{
stepnumber=0,
}
\tikzset{every tree node/.style={minimum width=2em,draw,circle},
         blank/.style={draw=none},
         edge from parent/.style=
         {draw, edge from parent path={(\tikzparentnode) -- (\tikzchildnode)}},
         level distance=1.5cm}

\begin{document}
%%%%%%%%%%%%%%%%%%%%%%%%%%%%%%%%%%%%%%%%%%%%%%%
\psetheader{Extra Practice 8}{}
%%%%%%%%%%%%%%%%%%%%%%%%%%%%%%%%%%%%%%%%%%%%%%%

\medskip

\section{Question 1}
In a game of "Among Us", there are two groups of characters - normal crewmates and
impostors. \\
\begin{center}
\includegraphics[scale=0.7]{amogus.png}
\end{center}
The rules for our version are:
\begin{itemize}
    \item Everyone starts off without a location and has to move to a location first (we'll assume
    this is not going to be a problem)
    \item Only the impostors can kill crewmates, and can only kill a crewmate if they are in the
    same location
    \item If a crewmate dies, he becomes a ghost and can still move around like a usual crewmate
    and do tasks
    \item Both impostors and crewmate can report a dead body, but must be in the same location
    as the dead body
    \item Impostors can pretend to do tasks as well
\end{itemize}

Define a class \texttt{\bfseries Place} that is initialized with a name. It should have the following methods:
\begin{itemize}
\item \texttt{\bfseries get\_name()} that returns the name of the place
\item \texttt{\bfseries get\_people()} that returns the names of all the people (dead or alive) in the
location currently in a string \texttt{\bfseries "[Person A], [Person B], [Person C] in
[name of place]"}. If there is no one, return \texttt{\bfseries "Nobody here"}
\end{itemize}

Define another class \texttt{\bfseries Person} that is initialized with a name. It should have the following
methods:
\begin{itemize}
\item \texttt{\bfseries get\_name()} that returns the name of the place
\item \texttt{\bfseries get\_state()} that returns \texttt{\bfseries "Alive"} if the person is still alive and 
\texttt{\bfseries "Killed by [Murderer's name]"} if he/she is dead
\item \texttt{\bfseries do\_task(task)} that takes in a task as an input string and returns the following
sentence \texttt{\bfseries "[Name of person] does [Name of task]."}
\item \texttt{\bfseries move(location)} that takes in a place object and moves the person from his current
place to that new location
\begin{itemize}
\item At the start of the game when this is called, return \texttt{\bfseries "[Name of person]
moves to [Name of place]"}
\item If the location is his current location, return \texttt{\bfseries "Already here"}
\item Otherwise, return \texttt{\bfseries "[Name of person] moves from [Current location name] to [New location name]"}
\end{itemize}
\item \texttt{\bfseries report(person)} that takes in a person object and tries to report it
\begin{itemize}
\item If the person reporting is already dead, return \texttt{\bfseries "Ghosts cannot report"}
\item If the person he is trying to report is oneself, return \texttt{\bfseries "Cannot report oneself"}
\item If the person he is trying to report is not in the same location as he is in, return
\texttt{\bfseries "[Name of other person] is in [Name of location]"}
\item If the person he is trying to report is still alive, return \texttt{\bfseries "[Name of person]
is still alive"}
\item Otherwise, return \texttt{\bfseries "[Name of person] reports [Name of other
person]..."} \\
He also immediately suspects everyone alive in the room at that moment and additionally returns the sentence 
\texttt{\bfseries "... and suspects [Person A's name], [Person B's name] ... [Person D's name]"} \\
If there is nobody else that he can suspect, the sentence \texttt{\bfseries "... and
nobody to suspect"} is returned
\end{itemize}
\end{itemize}

We now want to define another class \texttt{\bfseries Impostor} that is a subclass of \texttt{\bfseries Person}, and is
initialized with two inputs - name and weapon. It should have the following additional methods:
\begin{itemize}
\item \texttt{\bfseries kill(person)} that takes in a person object and attempts to kill it
\begin{itemize}
\item If the person is oneself, return \texttt{\bfseries "Cannot kill oneself"}
\item If the person is not in the same room as him, return \texttt{\bfseries "[Victim's name] is
in [Name of location victim is in]. Cannot kill"}
\item If the person is a fellow impostor, return \texttt{\bfseries "Cannot kill another impostor"}
\item If the person is already dead, return \texttt{\bfseries "Already killed"}
\item Otherwise, kill that person and return \texttt{\bfseries "[Name] killed [name of victim] with a [name of weapon]"}
\end{itemize}
\item \texttt{\bfseries victim\_list()} that returns the names of all the people he killed in alphabetical
order \texttt{\bfseries "[Name] killed [Person A], [Person B] ... [Person D]."}
\begin{itemize}
\item If he has not killed anyone yet, return \texttt{\bfseries "Killed nobody"}
\end{itemize}
\end{itemize}

\newpage
\textbf{Sample Execution:}
\begin{python}
>>> ravn = Impostor("Ravn", "Pistol")
>>> brendan = Impostor("Brendan", "Knife")
>>> daryl = Person("Daryl")
>>> tze = Person("Tze Lynn")
>>> ryan = Person("Ryan")
>>> clifton = Person("Clifton")
>>> daniel = Person("Daniel")
>>> room = Place("classroom")
>>> toilet = Place("toilet")
>>> hall = Place("hall")

>>> daniel.get_state()
'Alive'
>>> daniel.move(toilet)
'Daniel moves to toilet'
>>> daryl.move(room)
'Daryl moves to classroom'
>>> tze.move(hall)
'Tze Lynn moves to hall'
>>> ryan.move(toilet)
'Ryan moves to toilet'
>>> brendan.move(toilet)
'Brendan moves to toilet'
>>> ravn.move(room)
'Ravn moves to classroom'
>>> clifton.move(room)
'Clifton moves to classroom'
>>> room.get_people()
'Daryl, Ravn, Clifton in classroom'
>>> ravn.kill(ravn)
'Cannot kill oneself'
>>> ravn.victim_list()
'Killed nobody'
>>> tze.do_task("wiring")
'Tze Lynn does wiring'
>>> brendan.kill(clifton)
'Clifton is in classroom. Cannot kill'
>>> ravn.kill(clifton)
'Ravn killed Clifton with a Pistol'
>>> ravn.report(tze)
'Tze Lynn is in hall'
>>> ryan.report(brendan)
'Brendan is still alive'
>>> daniel.move(room)
'Daniel moves from toilet to classroom'
>>> ravn.report(clifton)
'Ravn reports Clifton and suspects Daryl, Daniel'
>>> ravn.victim_list()
'Ravn killed Clifton'
>>> clifton.get_state()
'Killed by Ravn'
>>> brendan.kill(ryan)
'Brendan killed Ryan with a Knife'
>>> brendan.move(room)
'Brendan moves from toilet to classroom'
>>> brendan.kill(ravn)
'Cannot kill another impostor'
>>> brendan.do_task("swipe card")
'Brendan does swipe card'
>>> tze.move(toilet)
'Tze Lynn moves from hall to toilet'
>>> tze.report(ryan)
'Tze Lynn reports Ryan and nobody to suspect'
>>> hall.get_people()
'Nobody here'
>>> room.get_people()    # order does not matter here
'Daryl, Ravn, Clifton, Daniel, Brendan in classroom'
>>> brendan.kill(daniel)
'Brendan killed Daniel with a Knife'
>>> ravn.report(daniel)
'Ravn reports Daniel and suspects Daryl, Brendan'
>>> brendan.victim_list()
'Brendan killed Daniel, Ryan'   # must be in alphabetical order
>>> brendan.kill(daniel)
'Already killed'
\end{python}

\section{Question 2}
\textbf{Sample Execution (a very long one):}
\begin{python}
general = Channel("general-helpdesk", ["Owner", "Tutor", "Student"])
announcements = Channel("announcements", ["Owner", "Tutor"])
secret = Channel("foobar", ["Owner"])

russell = Owner("Russell")
clifton = Tutor("Clifton")
aeron = Student("Aeron")
kenghwee = User("Keng Hwee")

def display_hall_of_mute():
    channels = [general, announcements, secret]
    print()
    print("HALL OF MUTE")
    for channel in channels:
        print(f"{channel.get_name()}: {channel.hall_of_mute()}")
    print()

print(russell.get_role())               # Russell is the owner!
print(clifton.get_role())               # Clifton is a Tutor
print(aeron.get_role())                 # Aeron is a Student
print(kenghwee.get_role())              # Keng Hwee has no role
print()
print(russell.join(general))            # Russell joins #general-helpdesk
print(russell.join(general))            # Russell has already joined #general-helpdesk
print(clifton.join(general))            # Clifton joins #general-helpdesk
print(russell.mute(aeron, None, None))  # Russell muted Aeron indefinitely. Reason: None
print(aeron.join(general))              # Aeron is muted!
                                        # therefore cannot join
print(russell.unmute(aeron))            # Russell unmuted Aeron!
print(aeron.join(general))              # Aeron joins #general-helpdesk
print(kenghwee.join(general))           # Keng Hwee has no permission to join #general-helpdesk
print(general.get_members())            # ['Aeron', 'Clifton', 'Russell']
print()
print(russell.join(announcements))      # Russell joins #announcements
print(clifton.join(announcements))      # Clifton joins #announcements
print(aeron.join(announcements))        # Aeron has no permission to join #announcements
print(kenghwee.join(announcements))     # Keng Hwee has no permission to join #announcements
print(announcements.get_members())      # ['Clifton', 'Russell']
print()
print(russell.join(secret))             # Russell joins #foobar
print(clifton.join(secret))             # Clifton has no permission to join #foobar
print(aeron.join(secret))               # Aeron has no permission to join #foobar
print(kenghwee.join(secret))            # Keng Hwee has no permission to join #foobar
print(secret.get_members())             # ['Russell']
print()
print(russell.get_channels())           # ['announcements', 'foobar', 'general-helpdesk']
print(clifton.get_channels())           # ['announcements', 'general-helpdesk']
print(aeron.get_channels())             # ['general-helpdesk']
print(kenghwee.get_channels())          # []
print()

print(russell.message(announcements, "Tutorial is canceled!"))          # #announcements --- Russell: Tutorial is canceled!
print(clifton.message(general, "Hooray!"))                              # #general-helpdesk --- Clifton: Hooray!
print(aeron.message(announcements, "Tutorial is canceled!"))            # Aeron has not joined #announcements
print(kenghwee.message(announcements, "Tutorial is canceled!"))         # Keng Hwee has not joined #announcements
print(russell.message(secret, "I am alone!"))                           # #foobar --- Russell: I am alone!
print(clifton.message(secret, "WHAT"))                                  # Clifton has not joined #foobar
print()

print(russell.mute(kenghwee, None, "Testing"))                          # Russell muted Keng Hwee indefinitely. Reason: Testing
print(clifton.mute(russell, 10, "Revenge"))                             # Cannot mute a fellow tutor
print(kenghwee.mute(russell, 100, "Revenge"))                           # Keng Hwee is not allowed to send messages!
                                                                        # because he's muted
print(kenghwee.message(announcements, "Tutorial is canceled!"))         # Keng Hwee has not joined #announcements
print(clifton.unmute(kenghwee))                                         # Clifton unmuted Keng Hwee!
print(kenghwee.mute(russell, 100, "Revenge"))                           # Keng Hwee doesn't have a permission to mute another user
print(aeron.mute(clifton, 3, None))                                     # Aeron doesn't have a permission to mute another user
print(clifton.mute(aeron, 5, "Why would you try to mute me?"))          # Clifton muted Aeron for 5 minutes. Reason: Why would you try to mute me?
print(kenghwee.mute(kenghwee, 10, "No idea"))                           # Cannot mute oneself
print(russell.mute(russell, 10, "Same here"))                           # Cannot mute oneself
print(russell.mute(aeron, 3, "Spam"))                                   # Aeron is muted!
display_hall_of_mute()
'''
HALL OF MUTE
#general-helpdesk: ['Aeron']
#announcements: []
#foobar: []
'''

print(aeron.message(secret, "Hello"))                                   # Aeron has not joined #foobar
print(aeron.message(general, "Yoooo"))                                  # Aeron is not allowed to send messages in #general-helpdesk
print(aeron.message(announcements, "Test"))                             # Aeron has not joined #announcements
print(clifton.mute(russell, None, None))                                # Cannot mute a fellow tutor
print(russell.mute(clifton, None, "Muting a fellow tutor is a can"))    # Russell muted Clifton indefinitely. Reason: Muting a fellow tutor is a can
display_hall_of_mute()
'''
HALL OF MUTE
#general-helpdesk: ['Aeron', 'Clifton']
#announcements: ['Clifton']
#foobar: []
'''

print(clifton.mute(aeron, 3, "Spam?"))      # Clifton is not allowed to send messages!
print(russell.unmute(clifton))              # Russell unmuted Clifton!
display_hall_of_mute()
'''
HALL OF MUTE
#general-helpdesk: ['Aeron']
#announcements: []
#foobar: []
'''

print(clifton.mute(aeron, 3, "Spam?"))      # Aeron is muted!
print(aeron.message(general, "Yoooo"))      # Aeron is not allowed to send messages in #general-helpdesk
                                            # since he's muted
print(aeron.mute(kenghwee, 2, "Lol"))       # Aeron is not allowed to send messages!
                                            # again, because he's still muted
print(aeron.unmute(kenghwee))               # Keng Hwee is not muted :)
print(clifton.unmute(aeron))                # Clifton unmuted Aeron!
display_hall_of_mute()
'''
HALL OF MUTE
#general-helpdesk: []
#announcements: []
#foobar: []
'''

print(kenghwee.mute(russell, 10, None))                     # Keng Hwee doesn't have a permission to mute another user
print(kenghwee.message(general, "Hi guys I'm unmuted"))     # Keng Hwee has not joined #general-helpdesk
print(russell.message(general, "Hello"))                    # #general-helpdesk --- Russell: Hello
print(clifton.message(general, "Hello!"))                   # #general-helpdesk --- Clifton: Hello!
print(aeron.message(general, "I'm so happy!"))              # #general-helpdesk --- Aeron: I'm so happy!
print(aeron.message(announcements, "Test"))                 # Aeron has not joined #announcements
print(kenghwee.join(general))                               # Keng Hwee has no permission to join #general-helpdesk
print(russell.unmute(kenghwee))                             # Keng Hwee is not muted :)
print(russell.unmute(russell))                              # Russell is not muted :)
\end{python}
\end{document}