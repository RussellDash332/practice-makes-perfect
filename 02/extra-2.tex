\input ../setup

\usepackage{hyperref}
\usepackage{listings}
\usepackage{color}
\usepackage{enumerate}
\usepackage{graphicx}
\usepackage{tikz-qtree}

% \renewcommand{\labelenumi}{\alph{enumi}.}
% \renewcommand{\labelenumii}{(\roman{enumii})}

\definecolor{mygreen}{rgb}{0,0.6,0}
\definecolor{mygray}{rgb}{0.5,0.5,0.5}
\definecolor{mymauve}{rgb}{0.58,0,0.82}


\lstdefinestyle{normalPy}{
language=Python,				% the language of the code
basicstyle=\footnotesize,			% the size of the fonts that are used for the code
numbers=left,				% where to put the line-numbers; possible values are (none, left, right)
numberstyle=\color{mygray},		% the style that is used for the line-numbers
stepnumber=1,				% the step between two line-numbers. If it's 1, each line will be numbered
numbersep=5pt,				% how far the line-numbers are from the code
backgroundcolor=\color{white},		% choose the background color; you must add \usepackage{color} or \usepackage{xcolor}
showstringspaces=false,			% underline spaces within strings only
showspaces=false,				% show spaces everywhere adding particular underscores; it overrides 'showstringspaces'
showtabs=false,				% show tabs within strings adding particular underscores
frame=shadowbox,				% adds a frame around the code
tabsize=4,					% sets default tabsize to 4 spaces
captionpos=t,				% sets the caption-position to bottom
breaklines=true,				% sets automatic line breaking
breakatwhitespace=false,			% sets if automatic breaks should only happen at whitespace
commentstyle=\color{mygreen},    	% comment style
keepspaces=true,                 		% keeps spaces in text, useful for keeping indentation of code (possibly needs columns=flexible)
keywordstyle=\color{blue},       		% keyword style
}

\lstdefinestyle{consolePy}{
stepnumber=0,
}
\tikzset{every tree node/.style={minimum width=2em,draw,circle},
         blank/.style={draw=none},
         edge from parent/.style=
         {draw, edge from parent path={(\tikzparentnode) -- (\tikzchildnode)}},
         level distance=1.5cm}

\begin{document}
%%%%%%%%%%%%%%%%%%%%%%%%%%%%%%%%%%%%%%%%%%%%%%%
\psetheader{Extra Practice 2}{}
%%%%%%%%%%%%%%%%%%%%%%%%%%%%%%%%%%%%%%%%%%%%%%%

\medskip

\section{Question 1}
Trace the following code. \textbf{(4 marks)}
\begin{python}
result = 0
for i in range(5):
    result += 1
print(result)

result2 = 0
for i in range(1, 5):
    result2 += 1
print(result2)
\end{python}

\section{Question 2}
Trace the following code. \textbf{(5 marks)}
\begin{python}
result = 0
for i in range(1, 13, 3):
    if i % 2 == 0:
        i += 2
    else:
        result //= i
    result += i

print(result)
\end{python}

\newpage

\section{Question 3}
Trace the following code. \textbf{(5 marks)}
\begin{python}
a, b, c = "east", "easter", "easy"
a, b, c = c, a, b

if a < b:
    a, b = b, a
else:
    if b < c:
        a += b
b = a + c

print(a[:-1])
print(b[1:])
print(c[::-1])
\end{python}

\section{Question 4}
To play a game of bowling, we will store our results from each throw in an integer such
as 1459. In this game, we will only play with 9 pins. 1459 means 1 pin is struck in the
first shot, 4 pins in the second shot, 5 pins in the third shot and a strike in the last shot.
\begin{enumerate}[(a)]
\item Define a function \texttt{\bfseries score} that takes in an integer and returns the total score of the
game (1 pin = 1 point). Use an iterative approach. \textbf{(4 marks)} \\
\textbf{Sample Tests:}
\begin{python}
>>> score(1459)
19
>>> score(999)
27
\end{python}

\item What is the order of growth (in space and time) of your solution in part (a)? Explain
your answer. \textbf{(2 marks)}

\item Define a function \texttt{\bfseries score\_recursive} that does the same thing as in part a but in a
\textbf{recursive} manner. \textbf{(4 marks)}

\item What is the order of growth (in space and time) of your solution in part (c)? Explain
your answer. \textbf{(2 marks)}

\item Define a function \texttt{\bfseries strike\_count} and \texttt{\bfseries strike\_count\_recursive} that takes in
an integer and returns the total number of strikes in the game. Use iteration and recursion respectively. \textbf{(8 marks)} \\
\textbf{Sample Tests:}
\begin{python}
>>> strike_count(919)
2
>>> strike_count(1234560)
0
>>> strike_count(9999)
4
\end{python}

\item Now, each strike is going to be worth an extra 5 points each! Using your previously
defined functions, define a new function \texttt{\bfseries score\_improved} that takes in an integer and
returns the total score. \textbf{(4 marks)} \\
\textbf{Sample Tests:}
\begin{python}
>>> score_improved(919)
29
>>> score_improved(1234)
10
>>> score_improved(12349)
24
\end{python}
\end{enumerate}

\section{Question 5}
\begin{enumerate}[(a)]
\item We will define another maskify function to encrypt our password. Given a password
of any length, we want to mask all the characters with "*". Define a function \texttt{\bfseries maskify}
that takes in a password as a string and returns the new masked word. Use \textbf{iteration}. \textbf{(4 marks)} \\
\textbf{Sample Tests:}
\begin{python}
>>> maskify("password")
'********'
>>> maskify("121")
'***'
\end{python}

\item State the time and space complexity of your solution. \textbf{(2 marks)}

\item Do part (a) with recursion. \textbf{(4 marks)}

\item State the time and space complexity of your solution. \textbf{(2 marks)}

\item Now, we want to put an "*" sign in between all of the letters. Define a function \texttt{\bfseries slot}
that does this recursively. \textbf{(6 marks)} \\
\textbf{Sample Tests:}
\begin{python}
>>> slot("pass")
'p*a*s*s'
>>> slot("123")
'1*2*3'
\end{python}

\item We want to insert the "*" sign now into consecutive letters that are identical to each
other only. Define a function \texttt{\bfseries advanced\_slot} that can do this recursively. \textbf{(6 marks)} \\
\textbf{Sample Tests:}
\begin{python}
>>> advanced_slot("pass")
'pas*s'
>>> advanced_slot("aaaaba")
'a*a*a*aba'
\end{python}
\end{enumerate}

\newpage

\section{Question 6}
Trace the following code. \textbf{(4 marks)}
\begin{python}
def weird_sum(n):
    if n == 0:
        return 0 
    else:
        return n + weird_sum(n - 2)

print(weird_sum(5))
\end{python}

\section{Question 7}
Trace the following code. \textbf{(4 marks)}
\begin{python}
for i in range(5):
    print(i)
    i += i
\end{python}

\end{document}