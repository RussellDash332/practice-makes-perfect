\input ../setup

\usepackage{hyperref}
\usepackage{listings}
\usepackage{color}
\usepackage{enumerate}
\usepackage{graphicx}
\usepackage{tikz-qtree}
\usepackage{amsmath}
\usepackage{amssymb}

% \renewcommand{\labelenumi}{\alph{enumi}.}
% \renewcommand{\labelenumii}{(\roman{enumii})}

\definecolor{mygreen}{rgb}{0,0.6,0}
\definecolor{mygray}{rgb}{0.5,0.5,0.5}
\definecolor{mymauve}{rgb}{0.58,0,0.82}


\lstdefinestyle{normalPy}{
language=Python,				% the language of the code
basicstyle=\footnotesize,			% the size of the fonts that are used for the code
numbers=left,				% where to put the line-numbers; possible values are (none, left, right)
numberstyle=\color{mygray},		% the style that is used for the line-numbers
stepnumber=1,				% the step between two line-numbers. If it's 1, each line will be numbered
numbersep=5pt,				% how far the line-numbers are from the code
backgroundcolor=\color{white},		% choose the background color; you must add \usepackage{color} or \usepackage{xcolor}
showstringspaces=false,			% underline spaces within strings only
showspaces=false,				% show spaces everywhere adding particular underscores; it overrides 'showstringspaces'
showtabs=false,				% show tabs within strings adding particular underscores
frame=shadowbox,				% adds a frame around the code
tabsize=4,					% sets default tabsize to 4 spaces
captionpos=t,				% sets the caption-position to bottom
breaklines=true,				% sets automatic line breaking
breakatwhitespace=false,			% sets if automatic breaks should only happen at whitespace
commentstyle=\color{mygreen},    	% comment style
keepspaces=true,                 		% keeps spaces in text, useful for keeping indentation of code (possibly needs columns=flexible)
keywordstyle=\color{blue},       		% keyword style
}

\lstdefinestyle{consolePy}{
stepnumber=0,
}
\tikzset{every tree node/.style={minimum width=2em,draw,circle},
         blank/.style={draw=none},
         edge from parent/.style=
         {draw, edge from parent path={(\tikzparentnode) -- (\tikzchildnode)}},
         level distance=1.5cm}

\begin{document}
%%%%%%%%%%%%%%%%%%%%%%%%%%%%%%%%%%%%%%%%%%%%%%%
\psetheader{Extra Practice 2 Solutions}{}
%%%%%%%%%%%%%%%%%%%%%%%%%%%%%%%%%%%%%%%%%%%%%%%

\medskip

\section{Question 1}
Trace the following code. \textbf{(4 marks)}
\begin{python}
result = 0
for i in range(5):      # this means i will be set to 0, 1, 2, 3, 4
    result += 1
print(result)   # 5

result2 = 0
for i in range(1, 5):   # this means i will be set to 1, 2, 3, 4
    result2 += 1
print(result2)  # 4
\end{python}

\section{Question 2}
Trace the following code. \textbf{(5 marks)}
\begin{python}
result = 0
for i in range(1, 13, 3):   # i =   1   4   7   10
    if i % 2 == 0:          
        i += 2              # i =       6       12
    else:
        result //= i    # result =  0       1
    result += i         # result =  1   7   8   20

print(result)   # 20
\end{python}

\section{Question 3}
Trace the following code. \textbf{(5 marks)}
\begin{python}
a, b, c = "east", "easter", "easy"
a, b, c = c, a, b   # a = "easy", b = "east", c = "easter"

if a < b:           # No, "easy" > "east"
    a, b = b, a
else:               # Yes
    if b < c:       # Yes, "east" < "easter"
        a += b      # a is now "easyeast"
b = a + c           # b is now "easyeast" + "easter" = "easyeasteaster"
                    # c remains "easter"

print(a[:-1])   # easyeas
print(b[1:])    # asyeasteaster
print(c[::-1])  # retsae
\end{python}

\section{Question 4}
To play a game of bowling, we will store our results from each throw in an integer such
as 1459. In this game, we will only play with 9 pins. 1459 means 1 pin is struck in the
first shot, 4 pins in the second shot, 5 pins in the third shot and a strike in the last shot.
\begin{enumerate}[(a)]
\item Define a function \texttt{\bfseries score} that takes in an integer and returns the total score of the
game (1 pin = 1 point). Use an iterative approach. \textbf{(4 marks)} \\
\textbf{Sample Tests:}
\begin{python}
>>> score(1459)
19
>>> score(999)
27
\end{python}
\textbf{Solution:}
\begin{python}
# Using while loop
def score(pins):
    result = 0
    while pins > 0: # or while pins
        result += pins % 10
        pins //= 10
    return result

# Using for loop
def score(pins):
    result = 0
    for i in str(pins):
        result += int(i)
    return result
\end{python}

\item What is the order of growth (in space and time) of your solution in part (a)? Explain
your answer. \textbf{(2 marks)} \\
\textbf{Solution:} \\
\textit{O(n) time and O(1) space where n is the number of pins for both methods. This is because we iterate through 
each of the pin once and do a constant number of operations within, overall by using only two or three variables.}

\item Define a function \texttt{\bfseries score\_recursive} that does the same thing as in part a but in a
\textbf{recursive} manner. \textbf{(4 marks)} \\
\textbf{Solution:}
\begin{python}
def score(pins):
    if pins < 10:
        return pins
    return pins % 10 + score(pins // 10)
\end{python}

\item What is the order of growth (in space and time) of your solution in part (c)? Explain
your answer. \textbf{(2 marks)} \\
\textbf{Solution:} \\
\textit{O(n) time and O(n) space where n is the number of pins. There are approximately n deferred operations, 
making the space complexity to be O(n).}

\item Define a function \texttt{\bfseries strike\_count} and \texttt{\bfseries strike\_count\_recursive} that takes in
an integer and returns the total number of strikes in the game. Use iteration and recursion respectively. \textbf{(8 marks)} \\
\textbf{Sample Tests:}
\begin{python}
>>> strike_count(919)
2
>>> strike_count(1234560)
0
>>> strike_count(9999)
4
\end{python}
\textbf{Solution:}
\begin{python}
# Iteration using while loop
def strike_count(pins):
    result = 0
    while pins > 0: # or while pins
        result += int(pins % 10 == 9)
        pins //= 10
    return result

# Iteration using for loop
def strike_count(pins):
    result = 0
    for i in str(pins):
        result += int(i == "9")
    return result

# Recursion
def strike_count(pins):
    if pins < 10:
        return int(pins == 9)
    return int(pins % 10 == 9) + strike_count(pins // 10)

# Note:
# int(statement) is equivalent to 1 if statement is true and 0 otherwise.
# You may use this to save some lines.
\end{python}

\item Now, each strike is going to be worth an extra 5 points each! Using your previously
defined functions, define a new function \texttt{\bfseries score\_improved} that takes in an integer and
returns the total score. \textbf{(4 marks)} \\
\textbf{Sample Tests:}
\begin{python}
>>> score_improved(919)
29
>>> score_improved(1234)
10
>>> score_improved(12349)
24
\end{python}
\textbf{Solution:}
\begin{python}
def score_improved(pins):
    return score(pins) + 5 * strike_count(pins)
\end{python}
\end{enumerate}

\section{Question 5}
\begin{enumerate}[(a)]
\item We will define another maskify function to encrypt our password. Given a password
of any length, we want to mask all the characters with "*". Define a function \texttt{\bfseries maskify}
that takes in a password as a string and returns the new masked word. Use \textbf{iteration}. \textbf{(4 marks)} \\
\textbf{Sample Tests:}
\begin{python}
>>> maskify("password")
'********'
>>> maskify("121")
'***'
\end{python}
\textbf{Solution:}
\begin{python}
# Closed form is NOT allowed!
def maskify(word):
    return "*"*len(word)

# Use iteration instead
def maskify(word):
    answer = ""
    for i in word:
        answer += "*"
    return answer
\end{python}

\item State the time and space complexity of your solution. \textbf{(2 marks)} \\
\textbf{Solution:} \\
\textit{O(n**2) time complexity due to string concatenation and O(n) space complexity where n is the length of the word.}

\item Do part (a) with recursion. \textbf{(4 marks)} \\
\textbf{Solution:}
\begin{python}
def maskify(word):
    if word == "":
        return ""
    return "*" + maskify(word[1:])
    # or return maskify(word[:-1]) + "*"
\end{python}

\item State the time and space complexity of your solution. \textbf{(2 marks)} \\
\textbf{Solution:} \\
\textit{O(n**2) time complexity due to string concatenation and O(n**2) space complexity where n is the length of the word. \\
We can visualize the recursion tree with n levels. Note that on each level it will take up O(n) space due to string concatenation.}

\item Now, we want to put an "*" sign in between all of the letters. Define a function \texttt{\bfseries slot}
that does this recursively. \textbf{(6 marks)} \\
\textbf{Sample Tests:}
\begin{python}
>>> slot("pass")
'p*a*s*s'
>>> slot("123")
'1*2*3'
\end{python}
\textbf{Solution:}
\begin{python}
def slot(word):
    if len(word) <= 1: # not == 1 since word can be an empty string
        return word
    return word[0] + "*" + slot(word[1:])
\end{python}

\item We want to insert the "*" sign now into consecutive letters that are identical to each
other only. Define a function \texttt{\bfseries advanced\_slot} that can do this recursively. \textbf{(6 marks)} \\
\textbf{Sample Tests:}
\begin{python}
>>> advanced_slot("pass")
'pas*s'
>>> advanced_slot("aaaaba")
'a*a*a*aba'
\end{python}
\textbf{Solution:}
\begin{python}
def advanced_slot(word):
    if len(word) <= 1:
        return word
    if word[0] == word[1]:
        return word[0] + "*" + advanced_slot(word[1:])
    # else this is executed
    return word[0] + advanced_slot(word[1:])
\end{python}
\end{enumerate}

\section{Question 6}
Trace the following code. \textbf{(4 marks)}
\begin{python}
def weird_sum(n):
    if n == 0:
        return 0 
    else:
        return n + weird_sum(n - 2)

print(weird_sum(5))
# RecursionError will occur because the base case n == 0 is never met.
\end{python}

\section{Question 7}
Trace the following code. \textbf{(4 marks)}
\begin{python}
for i in range(5):
    print(i)
    i += i
# Although we are modifying i on the third line, note that i is always
# preset back to the original sequence 0, 1, 2, 3, 4.
# This means the numbers 0, 1, 2, 3, 4 will be printed but on the end
# of the for loop, we will get i = 4 + 4 = 8.
\end{python}

\section{Question 8}
You might have known about Fibonacci numbers before. But, have you known about Lucas numbers? \\ \\
According to Wikipedia, we define Lucas numbers as follows.
\[
    L_n =
    \begin{cases}
        2 & \text{if } n = 0 \\
        1 & \text{if } n = 1 \\
        L_{n-1} + L_{n-2} & \text{if } n > 1
    \end{cases}
\]
\begin{enumerate}[(a)]
\item Define a \textbf{recursive} function \texttt{\bfseries lucas} that takes in a \textit{nonnegative integer} n and
returns $L_n$. \textbf{(2 marks)} \\
\textbf{Sample Tests:}
\begin{python}
>>> lucas(1)
1
>>> lucas(10)
123
\end{python}
\textbf{Solution:}
\begin{python}
def lucas(n):
    if n == 0:
        return 2
    elif n == 1:
        return 1
    return lucas(n-1) + lucas(n-2)
\end{python}

\item Do part (a) \textbf{iteratively}. \textbf{(3 marks)} \\
\textbf{Solution:}
\begin{python}
def lucas(n):
    a, b = 2, 1
    for _ in range(n):
        a, b = b, a + b
    return a
\end{python}

\item \textbf{(Optional)} Define a function \texttt{\bfseries lucas2} that takes in a \textit{positive integer} n and returns
\[\frac{L_{n-1}+L_{n+1}}{5}\]
Try to call this function for $n = 1, 2, \ldots, 10$. Does \texttt{\bfseries lucas2} seem familiar to you? \\
\textbf{Solution:}
\begin{python}
def lucas2(n):
    # Using float division is okay but lucas2 is actually
    # a sequence of integers!
    return (lucas(n - 1) + lucas(n + 1)) // 5

print(lucas2(1))    # 1
print(lucas2(2))    # 1
print(lucas2(3))    # 2
print(lucas2(4))    # 3
print(lucas2(5))    # 5
print(lucas2(6))    # 8
print(lucas2(7))    # 13
print(lucas2(8))    # 21
print(lucas2(9))    # 34
print(lucas2(10))   # 55
# lucas2 is actually the Fibonacci sequence!
\end{python}
For more information, you may read at \href{https://en.wikipedia.org/wiki/Lucas_number#Relationship_to_Fibonacci_numbers}{\underline{this Wikipedia article}}.
\end{enumerate}

\begin{flushright}
\vspace{2 cm}\textbf{\textit{Solution compiled by Russell Saerang.}}
\end{flushright}
\end{document}