\input ../setup

\usepackage{hyperref}
\usepackage{listings}
\usepackage{color}
\usepackage{enumerate}
\usepackage{graphicx}
\usepackage{tikz-qtree}
\usepackage{amsmath}
\usepackage{amssymb}

% \renewcommand{\labelenumi}{\alph{enumi}.}
% \renewcommand{\labelenumii}{(\roman{enumii})}

\definecolor{mygreen}{rgb}{0,0.6,0}
\definecolor{mygray}{rgb}{0.5,0.5,0.5}
\definecolor{mymauve}{rgb}{0.58,0,0.82}


\lstdefinestyle{normalPy}{
language=Python,				% the language of the code
basicstyle=\footnotesize,			% the size of the fonts that are used for the code
numbers=left,				% where to put the line-numbers; possible values are (none, left, right)
numberstyle=\color{mygray},		% the style that is used for the line-numbers
stepnumber=1,				% the step between two line-numbers. If it's 1, each line will be numbered
numbersep=5pt,				% how far the line-numbers are from the code
backgroundcolor=\color{white},		% choose the background color; you must add \usepackage{color} or \usepackage{xcolor}
showstringspaces=false,			% underline spaces within strings only
showspaces=false,				% show spaces everywhere adding particular underscores; it overrides 'showstringspaces'
showtabs=false,				% show tabs within strings adding particular underscores
frame=shadowbox,				% adds a frame around the code
tabsize=4,					% sets default tabsize to 4 spaces
captionpos=t,				% sets the caption-position to bottom
breaklines=true,				% sets automatic line breaking
breakatwhitespace=false,			% sets if automatic breaks should only happen at whitespace
commentstyle=\color{mygreen},    	% comment style
keepspaces=true,                 		% keeps spaces in text, useful for keeping indentation of code (possibly needs columns=flexible)
keywordstyle=\color{blue},       		% keyword style
}

\lstdefinestyle{consolePy}{
stepnumber=0,
}
\tikzset{every tree node/.style={minimum width=2em,draw,circle},
         blank/.style={draw=none},
         edge from parent/.style=
         {draw, edge from parent path={(\tikzparentnode) -- (\tikzchildnode)}},
         level distance=1.5cm}

\begin{document}
%%%%%%%%%%%%%%%%%%%%%%%%%%%%%%%%%%%%%%%%%%%%%%%
\psetheader{Extra Practice 9}{}
%%%%%%%%%%%%%%%%%%%%%%%%%%%%%%%%%%%%%%%%%%%%%%%

\medskip

All the best for your finals!
\section{Code Tracing}
\begin{enumerate}[(a)]
\item
\begin{python}
# CS1010S AY17/18 Sem 2 Finals
def foo(x):
    return lambda y: bar(x) if y % 2 else x
def bar(y):
    return lambda x: foo(x) if y % 2 else y

print(foo(2)(3)(4))
\end{python}

\item
\begin{python}
# CS1010S AY17/18 Sem 2 Finals
a = [0, 1, 2]
a.append(a)
b = [a[0] + a[1], a[1:2], a[3][3][2]]
print(b)
\end{python}

\item
\begin{python}
# CS1010S AY17/18 Sem 1 Finals
s = 'Lollapalooza'
d = {}
for i in range(len(s)):
    d[s[i % 5]] = s[i]
print(d)
\end{python}

\item
\begin{python}
# CS1010S AY17/18 Sem 2 Finals
def force(x):
    try:
        return int(x)
    except ValueError:
        return float(x)
    except Exception:
        return "NaN"

print(force("100"))
print(force("1.0"))
print(force("abc"))
\end{python}

\newpage
\item
\begin{python}
# CS1010S AY19/20 Sem 1 Finals
def foo(x):
    def baz(y):
        return lambda z: (x, y)[z]
    return lambda x: baz(x)

print(foo(-1)(0)(1))
\end{python}

\item
\begin{python}
# CS1010S AY19/20 Sem 2 Finals
lst = [[1], [2, 2], [3, 3, 3]]

def f(lst):
    for i in lst.copy():
        if len(i) < 2:
            i.append(1)
        if sum(i) < 5:
            i.pop()
        else:
            lst.extend(i)
        print(lst)
    return lst

print(lst is f(lst))
print(lst)
\end{python}

\item
\begin{python}
# CS1010S AY20/21 Sem 1 Finals
def wow(n):
    print(n)
    return lambda m: n + m

def twice(t):
    print('yes')
    return lambda x: t(t(x))

once = twice(twice)(wow(2))
print(once(1))
\end{python}

\item
\begin{python}
# CS1010FC AY14/15 Special Term I Finals
a = {1: 2, 2: 4, 3: 6, 4: 7}
for k in a:
    if k % 2 == 1:
        del a[k]
print(a)
\end{python}

\item
\begin{python}
# CS1010X AY16/17 Special Term I Finals
a = {(1, 2): 3, (3, 4): 5}
for k, v in a.items():
    a[[v, k[0]]] = k[1]
b = list(a.values())
b.sort(reverse = True)
print(b)
\end{python}

\item
\begin{python}
# CS1010S AY17/18 Sem 2 Midterm
def foo(y):
    return lambda x: x(x(y))
def bar(x):
    return lambda y: x(y)
print((bar)(bar)(foo)(2)(lambda x: x + 1))
\end{python}

\item
\begin{python}
# CS1010S AY19/20 Sem 1 Midterm
def foo(x):
    return x(lambda a: a + 1)
def kung(x):
    return foo(lambda a: a(x))
print(kung(foo)(9000))
\end{python}
\end{enumerate}

\section{Robbing a House}
A thief wants to rob a series of houses, but unfortunately, the houses are somehow
linked by an alarm system such that he cannot rob two houses side-by-side. Given a list
of houses containing the amount of money (in millions) in each house, return the
maximum amount the thief can earn. \\ \\
\textbf{Sample Tests:}
\begin{python}
>>> rob([3, 1, 4, 10, 2, 2, 9, 8])
23  # 3+10+2+8 = 23
>>> rob([1, 100, 99, 1, 3])
103 # 1+99+3 = 103
\end{python}

\newpage

\section{Number Sum Mania}
\textbf{\textit{(CS1010FC AY14/15 Special Term I Finals)}} \\ \\
A positive integer $n \ge 2$ can be expressed as the sum of a number of positive integers smaller
than $n$. For example,
\begin{align*}
2 &= 1+1 \\
3 &= 1+2 \\
&= 1+1+1 \\
4 &= 1+3 \\
&= 2+2 \\
&= 1+1+2 \\
&= 1+1+1+1 \\
5 &= 1+4 \\
&= 1+1+3 \\
&= 2+3 \\
&= 1+2+2 \\
&= 1+1+1+2 \\
&= 1+1+1+1+1
\end{align*}
The function \texttt{\bfseries num\_sum} returns the number of ways that an integer can be expressed as the sum
of a number of positive integers. From the above examples, it should be clear that:
\begin{python}
num_sum(2) = 1
num_sum(3) = 2
num_sum(4) = 4
num_sum(5) = 6
\end{python}

\begin{enumerate}[(a)]
\item Write the function \texttt{\bfseries num\_sum}. \textbf{BIG HINT:} \texttt{\bfseries num\_sum} is extremely 
similar to \texttt{\bfseries count\_change} which was discussed in lecture.

\item Write the function \texttt{\bfseries sum\_set} that will return a list of the lists of possible number 
combinations for the integer sums. \textbf{Hint:} Think about how to modify the answer for Part (a). \\
\textbf{Sample Execution:}
\begin{python}
>>> sum_set(2)
[[1, 1]]

>>> sum_set(3)
[[1, 1, 1], [2, 1]]

>>> sum_set(4)
[[1, 1, 1, 1], [2, 1, 1], [2, 2], [3, 1]]

>>> sum_set(5)
[[1, 1, 1, 1, 1], [2, 1, 1, 1], [2, 2, 1], [3, 1, 1], [3, 2],
[4, 1]]

>>> sum_set(6)
[[1, 1, 1, 1, 1, 1], [2, 1, 1, 1, 1], [2, 2, 1, 1], [2, 2, 2],
[3, 1, 1, 1], [3, 2, 1], [3, 3], [4, 1, 1], [4, 2], [5, 1]]
\end{python}

\item Write the function \texttt{\bfseries sum\_set\_product} that will return a list of the products of the integer
sums produced by \texttt{\bfseries sum\_set}, i.e. multiply together the components of each integer sum. You can
assume that you have access to the function \texttt{\bfseries sum\_set} even if you cannot do Part (b). \\
\textbf{Sample Execution:}
\begin{python}
>>> sum_set_product(2) # 1x1
[1]

>>> sum_set_product(3) # 1x1x1 and 2x1
[1, 2]

>>> sum_set_product(4)
[1, 2, 3, 4]

>>> sum_set_product(5) # Note that 4x1 = 2x2x1 so 5 elements, not 6
[1, 2, 3, 4, 6]

>>> sum_set_product(6)
[1, 2, 3, 4, 5, 6, 8, 9]
\end{python}

\item Write the function \texttt{\bfseries has\_prime\_sum} that will return \texttt{\bfseries True} for an 
integer $n$ if it can be expressed as a sum of \textbf{\underline{2}} prime numbers, or \texttt{\bfseries False} 
otherwise. Assume that you have access to the function \texttt{\bfseries is\_prime} that will return 
\texttt{\bfseries True} if an integer is prime. \\
\textbf{Sample Execution:}
\begin{python}
>>> has_prime_sum(2)
False

>>> has_prime_sum(3)
False

>>> has_prime_sum(4)    # 2+2
True

>>> has_prime_sum(5)    # 2+3
True

>>> has_prime_sum(6)    # 3+3
True

>>> has_prime_sum(11)   # Not possible!
False
\end{python}
\end{enumerate}
\end{document}