\input ../setup

\usepackage{hyperref}
\usepackage{listings}
\usepackage{color}
\usepackage{enumerate}
\usepackage{graphicx}
\usepackage{tikz-qtree}
\usepackage{amsmath}
\usepackage{amssymb}

% \renewcommand{\labelenumi}{\alph{enumi}.}
% \renewcommand{\labelenumii}{(\roman{enumii})}

\definecolor{mygreen}{rgb}{0,0.6,0}
\definecolor{mygray}{rgb}{0.5,0.5,0.5}
\definecolor{mymauve}{rgb}{0.58,0,0.82}


\lstdefinestyle{normalPy}{
language=Python,				% the language of the code
basicstyle=\footnotesize,			% the size of the fonts that are used for the code
numbers=left,				% where to put the line-numbers; possible values are (none, left, right)
numberstyle=\color{mygray},		% the style that is used for the line-numbers
stepnumber=1,				% the step between two line-numbers. If it's 1, each line will be numbered
numbersep=5pt,				% how far the line-numbers are from the code
backgroundcolor=\color{white},		% choose the background color; you must add \usepackage{color} or \usepackage{xcolor}
showstringspaces=false,			% underline spaces within strings only
showspaces=false,				% show spaces everywhere adding particular underscores; it overrides 'showstringspaces'
showtabs=false,				% show tabs within strings adding particular underscores
frame=shadowbox,				% adds a frame around the code
tabsize=4,					% sets default tabsize to 4 spaces
captionpos=t,				% sets the caption-position to bottom
breaklines=true,				% sets automatic line breaking
breakatwhitespace=false,			% sets if automatic breaks should only happen at whitespace
commentstyle=\color{mygreen},    	% comment style
keepspaces=true,                 		% keeps spaces in text, useful for keeping indentation of code (possibly needs columns=flexible)
keywordstyle=\color{blue},       		% keyword style
}

\lstdefinestyle{consolePy}{
stepnumber=0,
}
\tikzset{every tree node/.style={minimum width=2em,draw,circle},
         blank/.style={draw=none},
         edge from parent/.style=
         {draw, edge from parent path={(\tikzparentnode) -- (\tikzchildnode)}},
         level distance=1.5cm}

\begin{document}
%%%%%%%%%%%%%%%%%%%%%%%%%%%%%%%%%%%%%%%%%%%%%%%
\psetheader{Extra Practice 6}{}
%%%%%%%%%%%%%%%%%%%%%%%%%%%%%%%%%%%%%%%%%%%%%%%

\medskip

For Questions 1-6, what are the output(s) of the code?
\section{Question 1}
\begin{python}
a = [1, 2, 0, 4, 5, 6]

for i in range(len(a)):
    if a[i] % 2 == 0:
        a.pop(i)
    print(a)
\end{python}

\section{Question 2}
\begin{python}
a = []

temp = [0]*4
for i in range(3):
    a.append(temp)

a[1][1] = 99
print(a)
\end{python}

\section{Question 3}
\begin{python}
a = [1, 2, 3]
b = (1, 2, 3, a, 4, 5)
print(b)
a.clear()
print(b)
a = [1]
print(b)
\end{python}

\section{Question 4}
\begin{python}
a = [1, 2]
a += [a]
print(a)
b = a.copy() # shallow copy vs deep copy
a[2] = 0
print(a)
print(b)
\end{python}

\section{Question 5}
\begin{python}
a = [1, 2, 3]
b = [4, a]
c = [a, b, 15, b[0], b[1]]

c[1][0] = 99
print(b)
print(c)

c[3] = 100
print(b)
print(c)

a[1] = 200
print(a)
print(b)
print(c)

b[1][2] = 300
print(a)
print(b)
print(c)

d = [4, c.copy()]
c[2] = 500
c[0][2] = 500
print(d)
\end{python}

\section{Question 6}
\begin{python}
# CS1010X AY18/19 Special Term I Finals
a = [4,3,2,1]
b = [1,4,3,2]
c = [2,1,4,3]
d = [3,2,1,4]
a[0], a[1], a[2] = a, c, b
b[0], b[1], b[2] = a, b, c
c[0], c[1], c[2] = c, b, d
d[0], d[1], d[2] = c, d, a
print(a[0][1][0][2][1][2][2][2][3])
\end{python}

\newpage

\section{Question 7: Minesweeper!}
In Minesweeper, we have some mines placed in a 2-D field (starting from row 0, column
0) and store the positions of these mines within a list. For instance, the grid below.
\begin{python}
grid = [["O","O","X","O"],  # row 0
        ["X","O","X","X"],
        ["O","X","O","O"]]  # row 2

bombs = [[0, 2], [1, 0], [1, 2], [1, 3], [2, 1]]
\end{python}
For the position at row 1 column 2, there are a total of 4 bombs in its vicinity. "Vicinity" in
minesweeper means the bombs in a 3-by-3 grid with respect to its position in the middle.
\begin{enumerate}[(a)]
\item Define a function \texttt{\bfseries grid\_to\_bombs} that takes in a grid as shown and returns the
bombs list, containing the positions of all the bombs found in the grid (marked as "X").
\textbf{Sample Test:}
\begin{python}
>>> grid_to_bombs(grid) == bombs
True
\end{python}

\item Now, define a function \texttt{\bfseries count\_bombs} that takes in a row and a column, as well as a
grid, and returns the number of bombs in the vicinity of that row and column. \\
\textbf{Sample Test:}
\begin{python}
>>> count_bombs(1, 2, grid)
4
\end{python}
\end{enumerate}
\end{document}