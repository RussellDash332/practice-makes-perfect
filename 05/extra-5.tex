\input ../setup

\usepackage{hyperref}
\usepackage{listings}
\usepackage{color}
\usepackage{enumerate}
\usepackage{graphicx}
\usepackage{tikz-qtree}
\usepackage{amsmath}
\usepackage{amssymb}

% \renewcommand{\labelenumi}{\alph{enumi}.}
% \renewcommand{\labelenumii}{(\roman{enumii})}

\definecolor{mygreen}{rgb}{0,0.6,0}
\definecolor{mygray}{rgb}{0.5,0.5,0.5}
\definecolor{mymauve}{rgb}{0.58,0,0.82}


\lstdefinestyle{normalPy}{
language=Python,				% the language of the code
basicstyle=\footnotesize,			% the size of the fonts that are used for the code
numbers=left,				% where to put the line-numbers; possible values are (none, left, right)
numberstyle=\color{mygray},		% the style that is used for the line-numbers
stepnumber=1,				% the step between two line-numbers. If it's 1, each line will be numbered
numbersep=5pt,				% how far the line-numbers are from the code
backgroundcolor=\color{white},		% choose the background color; you must add \usepackage{color} or \usepackage{xcolor}
showstringspaces=false,			% underline spaces within strings only
showspaces=false,				% show spaces everywhere adding particular underscores; it overrides 'showstringspaces'
showtabs=false,				% show tabs within strings adding particular underscores
frame=shadowbox,				% adds a frame around the code
tabsize=4,					% sets default tabsize to 4 spaces
captionpos=t,				% sets the caption-position to bottom
breaklines=true,				% sets automatic line breaking
breakatwhitespace=false,			% sets if automatic breaks should only happen at whitespace
commentstyle=\color{mygreen},    	% comment style
keepspaces=true,                 		% keeps spaces in text, useful for keeping indentation of code (possibly needs columns=flexible)
keywordstyle=\color{blue},       		% keyword style
}

\lstdefinestyle{consolePy}{
stepnumber=0,
}
\tikzset{every tree node/.style={minimum width=2em,draw,circle},
         blank/.style={draw=none},
         edge from parent/.style=
         {draw, edge from parent path={(\tikzparentnode) -- (\tikzchildnode)}},
         level distance=1.5cm}

\begin{document}
%%%%%%%%%%%%%%%%%%%%%%%%%%%%%%%%%%%%%%%%%%%%%%%
\psetheader{Extra Practice 5}{}
%%%%%%%%%%%%%%%%%%%%%%%%%%%%%%%%%%%%%%%%%%%%%%%

\medskip

\section{Important Note}
\begin{python}
>>> tuple(range(8))
(0, 1, 2, 3, 4, 5, 6, 7)

>>> a = (1, 2, 3, 4)
>>> b = tuple(map(lambda x: x*3, a))
>>> b
(3, 6, 9, 12)
>>> c = tuple(filter(lambda x: x % 2 == 0, a))
>>> c
(2, 4)

>>> d = ("abcd", "bcdghi", "ab")
>>> e = sorted(d)
>>> e
['ab', 'abcd', 'bcdghi'] # a list

>>> f = "apple"
>>> g = sorted(f)
>>> g
['a', 'e', 'l', 'p', 'p'] # also a list

# So, you can sort either a tuple or a string, but you get a list.
# You can sort a list as well and get a list in return.
\end{python}

\section{Question 1}
Using \texttt{\bfseries range}, \texttt{\bfseries map} and \texttt{\bfseries filter}, define the
following functions to get the following outputs. (You can use other techniques as well)
\begin{enumerate}[(a)]
\item Define a function \texttt{\bfseries f1} that takes in no inputs, and returns \texttt{(6, 5, 4, 3, 2, 1, 0)}.

\item Define a function \texttt{\bfseries f2} that takes in a tuple as an input, and transforms it in the following
manner.
\begin{python}
>>> f2((1, 2, 3, 4, 5, 6, 7, 8))
(1.5, 3.5, 5.5, 7.5)
>>> f2(tuple(range(2, 17, 3)))
(5.5, 11.5)
\end{python}

\newpage
\item Define a function \texttt{\bfseries f3} that takes in two inputs - a tuple consisting of strings, and a word,
and returns a new tuple that consists of strings in the original tuple that are an \textbf{anagram} of the
word. An anagram is a word that is obtained when another word is scrambled up. (for example, "dormitory" is an
anagram of "dirtyroom") \\
\textbf{Sample Tests:}
\begin{python}
>>> words = ("toilet", "dirtyroom", "dirtyyroom", "ormitoryd")
>>> f3(words, "dormitory")
('dirtyroom', 'ormitoryd')
>>> f3(words, "dirtyrooom")
()
\end{python}
\end{enumerate}

\section{Question 2}
You designed a robot that is supposed to move around the house and mop the floor as it
does. You want it to be highly efficient such that it does not ever move in a loop (i.e. its path of
motion will never intersect). Let’s assume that it can only move north, south, east and west
(indicated by "N", "S", "E", "W"). \\ \\
\textbf{Using tuples only}, implement a function, \texttt{\bfseries check\_loop} that
takes in a tuple consisting of directions it moves and returns \texttt{\bfseries True} if there exists a loop and
\texttt{\bfseries False} otherwise. \\ \\
\textbf{Sample Tests:}
\begin{python}
>>> check_loop(("N", "E", "E", "S", "W", "W", "S"))
True # there is a loop after the first 6 moves
>>> check_loop(tuple("NESESEENWNWS"))
True
\end{python}

\section{Question 3}
In the army, there exists a chain of command such that work will be allocated from a
superior to someone else directly under his chain of command until the person with the lowest
rank has to do all the work. We will assume that there is strictly a one-one pairing where every
superior will only have one person directly under his command, and there are multiple
superiors in the army. Given a tuple that indicates this hierarchy such as
\begin{python}
chain = (
    ("Civilian", "Major1"), ("Corporal1", "Recruit1"),
    ("Corporal2", "Private"), ("Major1", "Officer1"),
    ("Major2", "Officer2"), ("Officer1", "Sergeant1"),
    ("Sergeant1", "Corporal2"), ("Officer2", "CFC")
    )
\end{python}
To understand this tuple, the rank on the left in each tuple is \textbf{directly higher-ranked} than the
rank on the right. There are multiple hierarchies that do not intersect. \\ \\
Define a function \texttt{\bfseries taiji} that takes in the chain of command as a tuple, and a rank, and
returns the lowest ranking person under his command (directly or indirectly) so that he knows
who the work will eventually go to.
\newpage
\textbf{Sample Tests:}
\begin{python}
>>> taiji(chain, "Civilian")
'Private'
>>> taiji(chain, "Major2")
'CFC'
>>> taiji(chain, "CFC")
'CFC'
\end{python}
\begin{enumerate}[(a)]
\item Use recursion to solve it.
\item Use iteration to solve it.
\end{enumerate}

\section{Question 4}
\textbf{\textit{(CS1010S AY20/21 Sem 1 Mock Midterm)}} \\ \\
Flappy Bird is a mobile game, developed by Vietnam-based developer Dong Nguyen and published by .GEARS Studios,
a small, independent game developer also based in Vietnam. The game has a side-scrolling format and the player 
controls a bird, attempting to fly between the gaps of vertical green pipes. \\ \\
In this problem, you will solve some problems based loosely on the game. \\ \\
Suppose that the bird is originally at ground level \texttt{(height = 0)} and it needs to fly through a series of gaps 
at various heights. In each step, the bird can move either up one level, down one level, or stay at the same 
level. \\ \\
The function \texttt{\bfseries can\_clear(gaps)} takes in the tuple representing the heights of each gap and
returns \texttt{\bfseries True} if the bird can successfully clear pipes, or \texttt{\bfseries False} otherwise. \\ \\
\textbf{Sample Execution:}
\begin{python}
>>> can_clear((1, 2, 1, 1))
True
>>> can_clear((2, 1, 1))
False
>>> can_clear((1, 0, 1, 1, 0))
True
>>> can_clear((5,))
False
>>> can_clear((0, 1, 1, 2, 3, 4, 5))
True
>>> can_clear(())
True
\end{python}
\begin{enumerate}[(a)]
\item Provide an \underline{iterative} implementation of \texttt{\bfseries can\_clear}.

\item What is the order of growth in \underline{time} and in \underline{space} for the function you wrote in Part (a)?
Briefly explain your answer.

\item Provide a \underline{recursive} implementation of \texttt{\bfseries can\_clear}.

\item What is the order of growth in \underline{time} and in \underline{space} for the function you wrote in Part (c)?
Briefly explain your answer.

\item Flying is not easy. Flying up costs \underline{2} units of energy per level. Staying at the same level costs 
\underline{1} unit of energy. Luckily, flying down costs \underline{no} energy per level. \\ \\
Implement the function \texttt{\bfseries energy\_cost(gaps)} that takes in a tuple of gap heights, and
returns the total number of units of energy that flying through the gaps will cost. \\
\textbf{Sample Execution:}
\begin{python}
>>> energy_cost((1, 2, 3))          # costs 2+2+2 = 6
6
>>> energy_cost((4,))               # costs 8
8
>>> energy_cost((1, 4, 4, 4, 0))    # costs 2+6+1+1+0 = 10
10
\end{python}

\item Next, suppose that the bird can still fly up or down \underline{only one level in each step}, but it now has
a \underline{limited amount of energy}. It takes 2 units of energy to fly up by one level, takes 1 unit of
energy to stay at the same level, and takes no energy to fly down by one level. The bird will fail
to clear a series of gaps if it is unable to reach the gap height or it does not have enough energy
to fly through the gaps. \\ \\
Implement the function \texttt{\bfseries enough\_energy\_to\_clear(energy, gaps)} that takes in an initial amount 
of energy (\textbf{int}) and a tuple of gap heights, and returns \texttt{\bfseries True} if the bird can successfully 
clear the pipes, or \texttt{\bfseries False} otherwise. \\
\textbf{Sample Execution:}
\begin{python}
>>> enough_energy_to_clear(6, (1,2,3))
True
>>> enough_energy_to_clear(10, (1,2,3))
True
>>> enough_energy_to_clear(5, (1,2,3))      # not enough energy
False
>>> enough_energy_to_clear(6, (1,2,3,0))
True
>>> enough_energy_to_clear(10, (1,2,3,4,4,4,3,2,1,0))
True
>>> enough_energy_to_clear(10, (1,4,4,4,0)) # gap difference too big
False
>>> enough_energy_to_clear(0, ())
True
\end{python}
\end{enumerate}
\end{document}