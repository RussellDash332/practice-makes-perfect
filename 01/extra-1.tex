\input ../setup

\usepackage{hyperref}
\usepackage{listings}
\usepackage{color}
\usepackage{enumerate}
\usepackage{graphicx}
\usepackage{tikz-qtree}

% \renewcommand{\labelenumi}{\alph{enumi}.}
% \renewcommand{\labelenumii}{(\roman{enumii})}

\definecolor{mygreen}{rgb}{0,0.6,0}
\definecolor{mygray}{rgb}{0.5,0.5,0.5}
\definecolor{mymauve}{rgb}{0.58,0,0.82}


\lstdefinestyle{normalPy}{
language=Python,				% the language of the code
basicstyle=\footnotesize,			% the size of the fonts that are used for the code
numbers=left,				% where to put the line-numbers; possible values are (none, left, right)
numberstyle=\color{mygray},		% the style that is used for the line-numbers
stepnumber=1,				% the step between two line-numbers. If it's 1, each line will be numbered
numbersep=5pt,				% how far the line-numbers are from the code
backgroundcolor=\color{white},		% choose the background color; you must add \usepackage{color} or \usepackage{xcolor}
showstringspaces=false,			% underline spaces within strings only
showspaces=false,				% show spaces everywhere adding particular underscores; it overrides 'showstringspaces'
showtabs=false,				% show tabs within strings adding particular underscores
frame=shadowbox,				% adds a frame around the code
tabsize=4,					% sets default tabsize to 4 spaces
captionpos=t,				% sets the caption-position to bottom
breaklines=true,				% sets automatic line breaking
breakatwhitespace=false,			% sets if automatic breaks should only happen at whitespace
commentstyle=\color{mygreen},    	% comment style
keepspaces=true,                 		% keeps spaces in text, useful for keeping indentation of code (possibly needs columns=flexible)
keywordstyle=\color{blue},       		% keyword style
}

\lstdefinestyle{consolePy}{
stepnumber=0,
}
\tikzset{every tree node/.style={minimum width=2em,draw,circle},
         blank/.style={draw=none},
         edge from parent/.style=
         {draw, edge from parent path={(\tikzparentnode) -- (\tikzchildnode)}},
         level distance=1.5cm}

\begin{document}
%%%%%%%%%%%%%%%%%%%%%%%%%%%%%%%%%%%%%%%%%%%%%%%
\psetheader{Extra Practice 1}{}
%%%%%%%%%%%%%%%%%%%%%%%%%%%%%%%%%%%%%%%%%%%%%%%

\medskip

\section{Question 1}
Let's start with a basic code tracing.
\begin{python}
x = 1

def foo(x):
    return x + 1

print(foo(2))
print(foo(x))
print(x)
\end{python}

\section{Question 2}
How about this one?
\begin{python}
x = 1

def add_one(x):
    print (x + 1)

print(add_one(2))
y = add_one(x)
print(add_one(y))
\end{python}

\section{Question 3}
Give the output of the following function call.
\begin{python}
x = 10
def ping(x):
    return pong(x + 4)
def pong(x):
    x += 1
    return x ** 2
  
print(ping(3))
print(pong(x))
ping(2) # what happens here?
\end{python}

\section{Question 4}
Give the output of the following function call.
\begin{python}
x, y = 1, 4
x, y = y, x

def ding(x):
    if x % 2 == 1:
        print("Alright")
    elif x % 3 == 1:
        print("Okay")
    if x ** 0.5 == y + 1:
        print("Awesome")
    else:
        print("This question sucks")

ding(x)
ding(y)
ding(9)
print(ding(3))  # what happens here?
\end{python}

\section{Question 5}
Give the output of the following function call.
\begin{python}
def check(word):
    if len(word) >= 3:
        print("gg")
    if word[0] == word[-1]:
        print("cool")
    elif word[::2] == "cdc":
        print("nice")
    else:
        print("end?")
    if word[3::-1] == "edoc":
        print("not yet")
    else:
        print("end now")
    
check("codec")
check("codecs")
check("ar")
\end{python}

\section{Question 6}
\begin{enumerate}[(a)]
\item Define a function named \texttt{\bfseries total\_legs} that takes in two inputs - the number of chickens and the number of cows,
and returns the total number of legs in total. \\
\textbf{Sample Execution:}
\begin{python}
>>> total_legs(2, 2)
12
>>> total_legs(1, 4)
18
>>> total_legs(0, 1)
4
\end{python}

\item A tax is imposed on a farm that charges the farmer \$2 for every leg present on the farm. Define a function named
\texttt{\bfseries tax\_count} that takes in two inputs - the number of chickens and the number of cows, and returns the amount of tax that
the farmer needs to pay. Use your previously-defined function(s). :) \\
\textbf{Sample Execution:}
\begin{python}
>>> tax_count(2, 2)
24
>>> tax_count(0, 2)
16
\end{python}
  
\item The farmer wants to see if the total number of animals he has on his farm exceeds 10. Otherwise, he needs to pay \$5
more as tax. Let’s define a function named \texttt{\bfseries too\_many} that takes in two inputs - the number of chickens and the number of cows,
and returns \texttt{\bfseries True/False} depending on whether the total number of animals exceeds 10. \\
\textbf{Sample Execution:}
\begin{python}
>>> too_many(2, 2)
False
>>> too_many(7, 4)
True
\end{python}

\item Now, we want to find the total amount that the farmer needs to pay in total as tax. Define a function named \texttt{\bfseries total}
that takes in the same two inputs and returns the amount he needs to pay. You need to use your previously-defined functions. \\
\textbf{Sample Execution:}
\begin{python}
>>> total(2, 2)
24
>>> total(10, 1)
53
\end{python}
\end{enumerate}

\section{Question 7}
Sometimes, we may wish to encrypt our password by adding some asterisks at the back of the word. We want to mask the final 4 characters
with "*". If the word is shorter than 4 letters, the entire word is masked. This function will be called \texttt{\bfseries maskify} that takes in a word and
returns the new masked word. \\
\textbf{Sample Execution:}
\begin{python}
>>> maskify("password")
'pass****'
>>> maskify("burger")
'bu****'
>>> maskify("cone")
'****'
>>> maskify("cs")
'**'
>>> maskify("cs1010s is fun")
'cs1010s is****'
\end{python}

\section{Question 8}
Last question! What does this function do?
\begin{python}
def iterate(x):
    total = 0
    for i in range(x):
        if x % 2 == 1:
            total += i
    return total
\end{python}

\end{document}